\RequirePackage{ifluatex}
\let\ifluatex\relax

\documentclass[aps,%
12pt,%
final,%
oneside,
onecolumn,%
musixtex, %
superscriptaddress,%
centertags]{article} %% 
\topmargin=-40pt
\textheight=650pt
\usepackage[english,russian]{babel}
\usepackage[utf8]{inputenc}
%всякие настройки по желанию%
\usepackage[colorlinks=true,linkcolor=black,unicode=true]{hyperref}
\usepackage{euscript}
\usepackage{supertabular}
\usepackage[pdftex]{graphicx}
\usepackage{amsthm,amssymb, amsmath}
\usepackage{textcomp}
\usepackage[noend]{algorithmic}
\usepackage[ruled]{algorithm}
\usepackage{lipsum}
\usepackage{indentfirst}
\usepackage{babel}
\usepackage{pgfplots}
\usepackage{setspace}
\linespread{1.2}
\pgfplotsset{compat=1.9}
\selectlanguage{russian}
\pgfplotsset{model/.style = {blue, samples = 100}}
\pgfplotsset{experiment/.style = {red}}
\theoremstyle{plain}
\binoppenalty=10000
\newtheorem{theorem}{Теорема}[section] %
\setlength{\parindent}{2.4em}
\setlength{\parskip}{0.1em}
\theoremstyle{definition}
\newtheorem{definition}{Определение}[subsection]
\theoremstyle{remark}
\newtheorem{remark}{Замечание}[section]

\newtheorem{corollary}{Следствие}
\newtheorem{proposition}{Proposition}
\newtheorem{example}{Пример}
\renewcommand*{\proofname}{Proof}

\newtheorem{lemma}{Лемма}[section]

\graphicspath{ {./image/} }
\usepackage{xcolor}
\usepackage{hyperref}


\begin{document}

\begin{titlepage} 
\begin{center}
\textbf{}\\[10.0cm]
\textbf{\LARGE Страхование и актуарная математика}\\[0.5cm]
\textbf{\Large Александр Широков ПМ-1701} \\[0.2cm]


\begin{center} \large
{Преподаватель:} \\[0.5cm]
\textsc {Радионов Андрей Владимирович}\\
\end{center}

\vfill 



{\large {Санкт-Петербург}} \par
{\large {2020 г., 7 семестр}}
\end{center} 
\end{titlepage}

% Table of contents
\begin{thebibliography}{3}
  \bibitem{A}
\end{thebibliography}
\tableofcontents
\newpage

\section{01.09.2020}

\subsection{О чём предмет}

Страхование вклада - если у банка возникают проблемы, то нам возвращают деньги банк в некоторой границе. Страхование - выход США из великой депрессии. Застраховать в принципе можно все что угодно.




\end{document}