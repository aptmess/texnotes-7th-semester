\documentclass[%
12pt, %
final, % 
oneside, % 
onecolumn, %  
centertags]{article} % относится к классу article и размер шрифта 12 пунктовб, {article: статья, report: отчеты и диссертации, book: книга, letter: письмо}

% ------ page construction 

\topmargin= -30pt % насколько сверху будет страница
\textheight= 650pt

% ------ Пакеты расширения

\usepackage[utf8]{inputenc} % задает кодировку, utf-8 кодировка, включающая в себя знаки почти всех языков мира
\usepackage[english, russian]{babel} % подключает необходимые языки, основным языком является английский
\selectlanguage{russian} % настройки будут на английском, но писать будет на русском

\usepackage{euscript}
\usepackage{supertabular}

\usepackage[colorlinks=true,linkcolor=black,unicode=true,urlcolor = blue]{hyperref} %hypered
\usepackage[pdftex]{graphicx} % для графики

\usepackage{amsthm, amssymb, amsmath, amsfonts} % математический пакет, математические шрифты
\usepackage{textcomp}
\usepackage[noend]{algorithmic}
\usepackage[ruled]{algorithm}
\usepackage{lipsum}
\usepackage{indentfirst}
\usepackage{babel}
\usepackage{pgfplots}
\usepackage{setspace}
\usepackage{xcolor}
\usepackage{hyperref}

\linespread{1.2} 
\setlength{\parindent}{2.4em}
\setlength{\parskip}{0.1em}

\pgfplotsset{compat=1.9}
\pgfplotsset{model/.style = {blue, samples = 100}} 
\pgfplotsset{experiment/.style = {red}}

\theoremstyle{plain}
\binoppenalty=10000

\newtheorem{theorem}{Теорема}[section] % theorem

\theoremstyle{definition}
\newtheorem{definition}{Определение}[subsection]

\theoremstyle{remark}
\newtheorem{remark}{Замечание}[section]

\newtheorem{corollary}{Следствие}

\newtheorem{solution}{Решение}

\newtheorem{proposition}{Proposition}

\newtheorem{example}{Пример}

\newtheorem{lemma}{Лемма}[section]

\renewcommand*{\proofname}{Proof}

\graphicspath{ {./image/} }


\begin{document}

\begin{titlepage} 
\begin{center}
\textbf{}\\[10.0cm]
\textbf{\LARGE Страхование и актуарная математика}\\[0.5cm]
\textbf{\Large Александр Широков ПМ-1701} \\[0.2cm]


\begin{center} \large
{Преподаватель:} \\[0.5cm]
\textsc {Радионов Андрей Владимирович}\\
\end{center}

\vfill 



{\large {Санкт-Петербург}} \par
{\large {2020 г., 7 семестр}}
\end{center} 
\end{titlepage}

% Table of contents
\begin{thebibliography}{3}
  \bibitem{A}
\end{thebibliography}
\tableofcontents
\newpage

\section{Конспекты лекций}

\subsection{Микроэкономические основы страхования}

\subsubsection{01.09.2020}

Страхование вклада - если у банка возникают проблемы, то нам возвращают деньги банк в некоторой границе. Страхование - выход США из великой депрессии. Застраховать в принципе можно все что угодно.

\subsubsection{03.09.2020}

В ситуациях неопределенностей человек принимает решение не на основании математического ожидания $E\xi$, а на основании математического ожидания некоторой функции полезности $Eu(\xi)$, где $u$ - некая функция полезности. За $w$ - обозначим капитал, а за $a$ - плата за риск, $\xi$ - потенциальные убытки. Тогда ситуация будет описываться:
$$Eu(w-\xi) \quad u(w-a)$$

Пусть $W=100$ и случайая величина убытков принимает следующие значения: $0$ с вероятностью $0.9$,$1$ с вероятностью $0.05$,$20$ с вероятностью $0.05$. Функция полезности - $u(x) = \ln(x+1)$. Математическое ожидание убытка $E\xi = 0.55$. Приходит банк и говорит продать за $60$.

$w-\xi$ - начальное состояние, $w-a$ - возможное состояние, сравнение полезностей

Посчитаем:
$$Eu(w-\xi) = E \ln(100-\xi+1) = \ln (101-0+1) \cdot 0.9 + \ln (100-1+1) \cdot 0.05+ \ln (100-10+1) \cdot 0.05 = 4.60941$$
$$E \ln(100+1-0.55) = 4.60966$$

Есть $W, u(\xi), \xi, f_{\xi}(x)$:
$$E\left(u(W-\xi)\right) = \int\limits_{-\infty}^{\infty}u(W-x)f_{\xi}(x)dx$$

Вместо бесконечностей используются границы интегрирования.
$$\Delta u = \frac{\Delta W}{W} \qquad du = \frac{d W}{W} \qquad u = \ln W$$

\begin{definition}
	Пусть есть набор случайных величин $\xi$ и будем задавать предпочтение подобным образом $\xi \geq \eta$ - предпочтение нестрого отношение, если существуют какие-то пары, которые находятся в бинарном отношении.

	Мы будем говорить про свойства отношений:
	\begin{enumerate}
		\item Пиолнота: $\xi \geq \eta$ или $\eta \geq \xi$
		\item Транзитивность: $\xi \geq \eta, \eta \geq \varepsilon$ $\Rightarrow$ $\xi \geq \varepsilon$
		\item Из первого следует рефлексиновть
	\end{enumerate}

	Будем говорить, что данное бинарное отношение является отношением \textit{эквивалентности} - рефлексивно, транзитивно, симметрично.
\end{definition}
\begin{definition}
	Будем говорить, что $\xi \geq \eta$ и $\xi \not \sim \eta$ - отношение строго порядка
\end{definition}
\begin{definition}
	$V: \Xi \to R$ - функция $V$ сохраняет упорядочивание, если $\xi \geq \eta$, то:
	$$V(\xi) \geq V(\eta)$$
\end{definition}
\begin{definition}
	Пусть есть набор $\mathbb{A}_j$. $B \in A$ является полным по упорядочиванию, если для любых элементов $a,b \in A$ существует элемент $\exists c \in B$, что либо $a \geq c > b$ либо $a >c \geq b$.
\end{definition}
\begin{theorem}
	На $\Xi \leq V$ существует отношение, сохраняющее отношение, тогда и только тогда, когда в $\Xi$ существует счетно или конечное подмножество плотное по упорядочиванию.
\end{theorem}

Как построить функцию полезности? Построим отношение порядка на множестве товаров, строим кривые безразличия - классы эквивалентности (все элементы внутри эквивалентны между собой), они не пересекаются.

Построим прямую, единичный вектор (бисскетриса). На что нужно умножить единичный вектор, чтобы попасть в точку пересечения, и высчитываем функцию полезности.

$$V^*(\xi) = Eu(\xi) = \sum\limits u(x_i)p_i = V(x_1)p_1+V(x_2)p_2$$
$$\xi: x_1 \mapsto p_1,  x_2 \mapsto p_2$$
Мы можем выбрать функцию полезности таким образом

\subsubsection{04.09.2020}

Пусть случайная величина принимает значения $x \mapsto p$ и $y \mapsto 1-p$. Введем обозначение для такой случайной величины:
$$(x,y)_p$$

Наложим некоторые ограничения:
\begin{itemize}
	\item Тогда индивид индеферентен: $(x,y)_1 \sim x$.

	Например: пусть $\xi$ равномерно распределен на отрезке $[0,1]$: $\xi \sim \mathbb{U}[0,1]$.
	\item $(x,y)_p \sim (y,x)_{1-p}$
	\item $((x,y)_p,y)_q \sim (x,y)_{pq}$

	Пример: $(1,(2,3)_{\frac{1}{2}})_{\frac{1}{2}}$ для игры $1 \mapsto \frac{1}{2},2 \mapsto \frac{1}{4},3 \mapsto \frac{1}{4}$
\end{itemize}

Ограничения:
\begin{itemize}
	\item $\{p \in [0,1]: (x,y)_p \geq z\}$ - замкнутное множество
	\item $\{p \in [0,1]: z \geq (x,y)_p \}$ - замкнутное множество $\forall x,y,z \in \Xi$
	\item $x \sim y: (x,z)_p \sim (y,z)_p$
	\item $\exists w,b: \forall \xi \in \Xi$ выполняется, что $w \geq x \geq b$
	\item $(b,w)_p \geq (b,w)_q \Leftrightarrow p > q$
\end{itemize}

\begin{theorem}
	Если $\Xi$ и на нём введено отношение предпочтения $\geq$, то найдётся такая функция $V$, что $$V((x,y)_p) = pV(x)+(1-p)V(y)$$
	$$Eu(\xi) = (x,y)_p$$
	$$V(\xi) = Eu(\xi)$$
	$$u(y): R \to R, V: \Xi \to R$$
\end{theorem}

Доказательство: Хэливэриан.

Рассмотрим некоторый функционал $V((x,y)_p)$ и применим к нему некоторое преобразование:
$$f\left(V((x,y)_p)\right) = f(V(x)\cdot p+V(y)\cdot (1-p)) = f(V(x))\cdot p + f(V(y))\cdot (1-p)$$

и это линейная функция - линейное преобразование.

Пример: $u(x) = \ln (x+1)$, $\xi, f_{\xi}(y), Eu(\xi) = \int\limits u(x)f_{\xi}(x)dx$

$u_1(x) = a\ln(x+1) + b, a>0$
$$Eu_1(\xi) =  \int\limits  a\ln(x+1) + bf_{\xi}(x)dx = a\int ln(x+1)f(x)dx+b\int\limits f(x)dx = a\int\limits u(x)f_{\xi}(x)dx + b$$
следовательно функционал является единственным с точностью до линейного преобразования.

\subsubsection{08.09.2020}

Напоминание:
$$V(\xi) = \sum\limits p_j V(x_j)$$
и при дискретных $\xi$:
$$V(x_j) = u(x_j)$$

Рассмотрим некоторые свойства, которыми должна обладать функция полезности:

\begin{enumerate}
	\item Функция начинается в нуле из-за монотонного преобразования
	\item Функция полезности $u(x)$ не убывает (возрастает):
	$$u'(x) > 0$$
	\item Функция $u(x)$ вогнутая. 
	$$u(\lambda x_1 +(1-\lambda)x_2) \geq \lambda u(x_1) + (1-\lambda)u(x_2)$$

	Заметим, что если функция полезности вогнутая, то находясь в ситуации неопределенности, индивид будет согласен заплатить, чем иметь состояние неопределенности. Человек хочет иметь детерменированный выигрыш, нежели при ситуации неопределенности - это происходит из-за вогнутости функции.
	\begin{itemize}
		\item есть функция вогнута, то говорят \textsc{risk aversion}
		\item если выпукла, то говорят \textsc{risk loving}
		\item если функция линейна, то \textsc{risk neutral}
	\end{itemize}

	Почитать \href{https://en.wikipedia.org/wiki/Risk_aversion}{здесь} можно.
	\item $$Eu(\xi) \leq u(E(\xi))$$
	
	Сравнивает полезность ситуации $u(w-a)$ - нет риска, чуть меньше денег, и есть риск и чуть больше денег - $Eu(w-\xi)$ и если больше, то он соглашается - страхование возможно для некоторого $a$ и человек готов заплатить. С помощью неравенства Йенсена:
	$$Eu(w-\xi)\leq u(E(w-\xi)) = u(w-E\xi)$$
	$$u(w-a) \geq Eu(w-\xi) \Leftrightarrow u(w-E\xi) = u(w-a)$$
	и следовательно мы сможем найти $a = E\xi$ из которого будет выполняться свойство.

	\begin{example}
		Возьмем экспоненциальную функцию полезности: $$u(x) = 1-e^{-\lambda x}$$

		При желании для любой ограниченной функции можно подобрать лотерею так, в котороый можно подбирать математическое ожидание, чтобы человек всегда играл.

		В данном случае у нас ограниченная функция и ограниченное математическое ожидание.
	\end{example}
	\begin{example}
		Степенная функция полезности: $$u(x)=x^{\alpha}, \alpha < 1$$
	\end{example}
	\begin{example}
		Квадратичная функция полезности:
		$$u(x) = bx - cx^2: b,c >0, x < \frac{b}{2c}$$
	\end{example}
\end{enumerate}

\textsc{Задача 1.}
Пусть есть инвестор с капиталом $w$ и он может вложить деньги в $2$ неколлериованных $\xi_1 \sim N(\mu_1,\sigma_1^2)$ и $\xi_2 \sim N(\mu_2,\sigma_2^2)$. $u(x) = 1 - e^{-\lambda x}$. $\xi_1,\xi_2$ - это доходность, которая выражена в процентах.  В какой пропорции нужно разделить капитал, чтобы максимизировать нашу полезность.

\textsc{Решение}

Введём доли $\alpha$ и $1-\alpha$. Тогда доход инвестора будет вычисляться по формуле:
$$s = \alpha w (1+\xi_1) + (1-\alpha)w(1+\xi_2)$$
Будем максимизировать математическое ожидание от функции полезности:
$$Eu(s) = E (1-e^{-\lambda \cdot s}) = 1 - E \left(e^{-\lambda \left(\alpha w (1+\xi_1) + (1-\alpha)w(1+\xi_2)\right)}\right) \to \underset{\alpha}{\max}$$

Раскроем скобки и упростим выражение:
$$1 - E \left(e^{-\lambda \left(\alpha w (1+\xi_1) + (1-\alpha)w(1+\xi_2)\right)}\right) = 1 - E \left(e^{-\lambda w \left(\alpha(1+\xi_1) + (1-\alpha)(1+\xi_2)\right)}\right) = $$
$$ = 1 - E\left(e^{-\lambda w (\alpha\xi_1 + 1 + \xi_2 - \alpha \xi_2)}\right) = 1 - e^{-\lambda w}E\left(e^{-\lambda w (\alpha \xi_1 + \xi_2 (1-\alpha))}\right) \to \max \Rightarrow$$
$$E\left(e^{-\lambda w (\alpha \xi_1 + \xi_2 (1-\alpha))}\right) \to \min$$

Так как величины неколлерированы, то:
$$Ee^{-\lambda w \alpha \xi_1} \cdot Ee^{-\lambda w \xi_2 (1-\alpha)} \to \min$$

Сделаем замену $\beta = -w\lambda\alpha$ и попытаемся взять следующий интеграл
$$Ee^{-w\lambda\alpha\xi_1} = Ee^{\beta \xi_1} = \int\limits_{-\infty}^{\infty} e^{\beta x_1}\frac{1}{\sqrt{2\pi}\sigma}e^{-\frac{(x_1-\mu_1)^2}{2\sigma_1^2}}dx_1$$

Известно, что производящая функция моментов для нормального распредления есть следующая величина:
$$Ee^{\beta \xi_1} = e^{\mu\beta + \frac{\beta^2 \sigma_1^2}{2}}$$

Тогда преобразуем выражение:
$$Ee^{-\lambda w \alpha \xi_1} \cdot Ee^{-\lambda w \xi_2 (1-\alpha)} = e^{\left(-\mu_1w\lambda\alpha + \frac{w^2 \lambda^2\alpha^2 \sigma_1^2}{2}-\mu_2w\lambda(1-\alpha) + \frac{w^2 \lambda^2(1-\alpha)^2 \sigma_2^2}{2}\right)} \to \underset{\alpha}{\min}$$
$$-w\lambda\mu_1 + w^2\lambda^2\alpha\sigma_1^2 + w\lambda\mu_2-w^2\lambda^2(1-\alpha)\sigma_2^2 = 0$$
$$\alpha = \frac{\mu_1-\mu_2+w\lambda\sigma_2^2}{w\lambda\sigma_1^2+w\lambda\sigma_2^2}$$

Пусть $\mu_1=\mu_2: \alpha = \frac{\sigma_2^2}{\sigma_1^2+\sigma_2^2}$. В какой пропорции нужно разделить, чтобы минимизировать дисперсию.
$$D(w\alpha(1+\xi_1))+D(w(1-\alpha)(1+\xi_2)) = w^2 \alpha^2 D\xi_1 + w^2 (1-\alpha)^2 D\xi_2 = w^2\alpha^2\sigma_1^2 + w^2(1-\alpha)^2\sigma_2^2$$
$$2\alpha w^2\sigma_1^2 -2w^2(1-\alpha)\sigma_2^2 = 0$$
$$\alpha = \frac{\sigma_2^2}{\sigma_1^2+\sigma_2^2}$$

Решение, максимизирующее полезность соответствует решению, минимизирующую диспесию портфеля. Совпадения есть в случае $\mu_1 = \mu_2$. Интересно посмотреть через призму полезности.

Активы, различные портфели, ожидание и дисперсия. $\mu,\sigma$ - спектр доходности. Почему он определяется выпуклой фигурой.

Стандартное отклонение - выпуклая функция. Если есть возможность выбрать из различных портфелей, то мы можем сформировать любой портфель. Данное множество - плотно , сплошное (говорим про овал). 

Такое множество называется \textsc{эффективное множество}. Набор не даёт конкректный портфель, потому что мы не понимаем в чём разница между портфелями. \textbf{Критерий выбора} - поиск точки, в которой полезность максимальная - на кривой выбирает тот портфель, который дает максимальная полезность и в этом случае будет достигаться баланс между двумя теориями.

\textsc{HT}: кривые безразличия - те портфели, между которыми клиент индеферентен. В осях $\mu,\sigma$ и если рассмотреть одинаково полезные портфели, то они будут образовывать выпуклую кривую. И тогда решение - это точка касательной множества всех портфелей и совпадать с базовыми теорями кривых безразличия.

\newpage
\subsection{HT1 Свойства функции полезности}

1. Определить, какую максимальную сумму агент с капиталом $100$ и функцией полезности $u(x) = 5x - 0.01x^2$ согласится заплатить, чтобы избавиться от потенциального ущерба, принимающего значения $0,10,20,30$ с равными вероятностями.

\begin{solution}
	Величина ушерба - случайная величина с данным (известным) распределением, обозначим за $\xi$.

	Величина $E\xi = \sum\limits_{i=1}^4 p_i\xi_i = 15$ - ожидаемая величина ущерба в следующий промежуток времени. $u(x)$ - функция полезности от капитала, а $a$ - величина, которую агент может заплатить, если хочет избавиться от риска.

	Необходимо сравнить две величины. Первая - $E(u(w-\xi))$ - ожидаемая полезность при отказе от платы. Вторая - $u(w-a)$ - ожидаемая полезность при выплате суммы $a$ за полный отказ от риска.

	Так как $u(w)'>0$, а $w(w)''<0$, то есть функция возрастает и вогнута, то по неравенству Йенсена:
	$$E(u(w-\xi)) \leq u(E(w-\xi)) = u(w - E\xi)$$

	Для того, чтобы найти максмальную сумму, которую агент согласится заплатить, необходимо приравнять ожидаемую полезность при отказе и ожидаемую полезность при выплате суммы $a$ и решить полученное равенство относительно $a$:
	$$E(u(w-\xi)) = u(w - a)$$
	$$E(u(w-\xi)) = \sum\limits_{i=1}^4 u(w-\xi_i)p_i = \sum\limits_{i=1}^4 p_i (5(w-\xi_i)-0.01\cdot (w-\xi_i)^2) = 351.5$$
	$$u(w - a) = 5 (100-a) - 0.01(100-a)^2 = 351.5$$
	$$a = 15.3784$$
\end{solution}

\newpage
2. Определить, при каком значении капитала агент из предыдущей задачи будет наиболее интересен страховой организации, а при каком - наименее интересен.

\begin{solution}
	В прошлой задаче мы определились, что максимальную величину агент готов будет заплатить при выполнении равенства:
	$$E(u(w-\xi)) = u(w - a)$$

	Агент будет наиболее интересен компании, когда $a \to \max$ (когда выплачивается агентом максимальное количество денег) и менее интересен, когда $a \to \min$.

	Идея: выразить $a$ через $w$ и найти максимум и минимум функции по $w$.

	Получим квадратное уравнение относительно $w$:
	$$0.01a^2 - a \cdot (0.02w+5) + 0.3w - 78.5 = 0$$
	$$a = -250 + w \mp \sqrt{70350-530w+w^2}$$

	Осталось выбрать, как ограничивать $u,a$ и $w$. $a$, наверное, не может быть меньше нуля, тогда это означает, что страховая компания должна заплатить. Тогда, в одном из решений, решая относительно $w$, получим, что $a_{min} = a(w_{min}) = a(261.667)$.

	Дальше стоит вопрос как ограничивать $u$ и $w$. Снизу есть ограничение по $w$: 0, так как капитал не может быть отрицтаельным. Что есть верхняя граница $w$? Два варианта: точка, в которой функция полезности начинает убывать, либо точка, в которой функция полезности равна нулю.

	Тогда ответы, соответственно, $w_{max} = 200$ или $w_{max} = 500$
\end{solution}

\newpage
3.1 Решить первую задачу в случае, если потенциальный ущерб определяется случайной величиной с плотностью распределения $f_{\xi}(x) = a\sqrt{25-x^2}, x \in [0;5]$, а функция полезности есть:
$u(x) = \ln x = \log_e x $ или $u(x) = \lg x = \log_{10} x$

\begin{solution}
	$$E(u(w-\xi)) = u(w - a)$$
	$$u(w-a) = \ln(100 - a)$$
	$$E(u(w-\xi)) = E\left(\ln(100-\xi)\right) = \int\limits_{0}^5 \ln(100 - x) a\sqrt{25-x^2}dx$$

	Нужно взять интеграл, если нечего будет делать, $a \approx 0.05$
\end{solution}

4. Инвестор хочет распределить свой капитал между ценной бумагой, доходность по которой определяется $\xi_1 \sim N(\mu_1,\sigma_1^2)$ с матемматическим ожиданием $5\%$ и стандартным отклонением $2\%$ и безрисковой ценной бумагой с фиксированной доходностью $4\%$. 

Какую часть своего капитала инвестору стоит вложить в первую ценную бумагу, если его функция полезности есть $u(x) = 1 - e^{-ax}$
\begin{solution}
	Введём доли $\alpha$ и $1-\alpha$. Тогда доход инвестора вычислим по формуле:
	$$ s = w\alpha(1+\xi_1) + 1.04 \cdot w(1-\alpha)$$

	Будем максимизировать математическое ожидание от функции полезности:
	$$Eu(s) = E (1-e^{-\lambda \cdot s}) = 1 - E \left(e^{-\lambda \left(w\alpha(1+\xi_1) + 1.04 \cdot w(1-\alpha)\right)}\right) \to \underset{\alpha}{\max}$$

	Раскроем скобки и упростим выражение:
	$$1 - e^{-\lambda w 1.04} \cdot Ee^{-\lambda w \alpha (\xi_1 - 0.04)} \to \underset{\alpha}{\max}$$
	$$Ee^{-\lambda w \alpha (\xi_1 - 0.04)} \to \underset{\alpha}{\min}$$

	Сделаем замену $\beta = -w\lambda\alpha$ и попытаемся взять следующий интеграл
	$$Ee^{-w\lambda\alpha\xi_1} = Ee^{\beta \xi_1} = \int\limits_{-\infty}^{\infty} e^{\beta x_1}\frac{1}{\sqrt{2\pi}\sigma}e^{-\frac{(x_1-\mu_1)^2}{2\sigma_1^2}}dx_1$$

	Известно, что производящая функция моментов для нормального распредления есть следующая величина:
	$$Ee^{\beta \xi_1} = e^{\mu\beta + \frac{\beta^2 \sigma_1^2}{2}}$$
	$$Ee^{-\lambda w \alpha \xi_1} = e^{-\mu_1w\lambda\alpha + \frac{w^2 \lambda^2\alpha^2 \sigma_1^2}{2}}$$
	$$-\mu_1w\lambda + w^2 \alpha \lambda^2 \sigma_1^2 = 0$$
	$$\alpha = \frac{\mu_1}{w\lambda\sigma_1^2} = \frac{5}{w\cdot a \cdot 4}$$


\end{solution}

5. Решить предыдущую задачу, если инвестор распределяет капитал между двумя ценными бумагами, доходности которых распределены нормально с математическими ожиданиями $\mu_1,\mu_2$, стандартными отклонениями $\sigma_1, \sigma_2$ и коэффициентов корреляции $\rho$.

$$cov(\xi,\eta) = \sqrt{D\xi \cdot D\eta} \cdot \rho(\xi,\eta)$$
$$E(\xi\eta) - E\xi\eta = \sqrt{D\xi \cdot D\eta} \cdot \rho(\xi,\eta)$$
$$E(\xi\eta) = \sqrt{D\xi \cdot D\eta} \cdot \rho(\xi,\eta) + E\xi\eta$$
$$D\xi = E\xi^2 - (E\xi)^2$$
$$E(\xi\eta) = \sqrt{\left(E\xi^2 - (E\xi)^2\right)\cdot \left(E\eta^2 - (E\eta)^2\right)}\cdot \rho(\xi,\eta) + E\xi\eta \to \underset{\alpha}{\min}$$

Нам известно всё, кроме $E\xi^2$


\subsection{}

Хотим понять, кто менее склонен к риску. Давайте предлагать игру с маленькими выигрышами, игра характеризуется маленькой дисперсией. Кто готов заплатить в этой игре больше, то тем меньше человек склонен к риску.

Давайте разложим правую и левую часть в ряды Тейлора в окрестности капитала $x_0 = w$

Не умоляя общности положим $E\xi = 0$. Чем коэффициент больше тем будет больше Risk Aversion. Данный коэффициент называется Эрроу-Пратт. Первая производная отрицательная, а вторая положительная.

Если коэффициент Эророу-Прата возрастает, то чем больше капитал, тем больше мы готовы к риску.

Такую экспоненциальную функцию называют Constant Avertion, Relative COnst Aversion

Каро-утилити. А

CARA

Аксиоматика задних чисел

\subsection{Страховой контракт}

Есть $w$, $\xi$ и $a$:
$$Eu(w-\xi) \leqslant u(w-a)$$

Страхователь - я, страховщик - они. Что странивает для себя страховщик. $u(w_1)$ - начальное состояние, а альтернатива $Eu(w_1+a-\xi)$
$$u(w_1) \leqslant Eu(w_1+a-\xi)$$

\textsc{Задача:} компания будет платить только половину убытка.

Страхование эксцедента 

\subsection{23.09.2020}

Капитал - $w$, риск - $\xi$, страховая премия - $a$, величина, которую вы получите при ущербе - $I(\xi)$:
$$Eu(w-\xi) \qquad E(u(w-a-(\xi - I(\xi))))$$

Если убыток большой, то остальную сумму заплатит страховая компания

Теорема Фон-Неймана-Моргенштерна

Люди в среднем выбираби чаще 2 чем 1 и 4 чем 3, но это противоречит предпосылке поведения теории, потому что если 2 и 4 лучше 1 и 3 (люди выбирают), то тогда они должны быть лучше, а вероятности однаковы.

\newpage
\section{Моделирование риска}

\textbf{Основные понятия}:

\subsection{Распределение индивидуального ущерба. Распределение условного и безусловного ущерба, их ожидания и дисперсии}

Под распределением ущерба понимают вероятностное распределение, увязывающее частоту возникновения и размер ущерба. Это наиболее простая модель, позволяющая количественно исследовать неопределенность величины ущерба в контексте
управления рисками.

Начнём с ситуации, когда во внимание принимается та информация, которая касается уже возникшего ущерба. Данная модель не учитывает: данные по объектам, по которым ущерб не возникал, неполнота сведений о возникновении ущерба.

Итого: ущерб - случайная величина. Если известна величина максимально возможного ущерба $M$, то распределение сосредоточено на отезке $[0;M]$.

Нас интересует статистика по тем объектам, по которым имел место ущерб, так и по тем носителям риска, по которым \textbf{его не было}. 

\begin{definition}
	Доля объектов в портфеле, которых не было, можно рассматривать как вероятность отсутствия ущерба по одному наугад выбранному риску.
\end{definition}

Поэтому будем использовать иную случайную величину $X$ в качестве более адекватной модели ущерба. Она будет с ненулевой вероятностью принимать $\text{0}$, отражающее \textbf{отсутствие ущерба}.

Пусть $Y$ - случайная величина размера ущерба. Введём дополнительную индикаторную величину $\mathbb{I}$:
$$\begin{cases}
	1  - \text{ущерб возник}\\
	0 - \text{нет}
\end{cases}$$

С помощью данной случайной величины моделируется неопределенность, связанная с возникновением ущерба - неопределенность числа неблагоприятных событий.

$$F_{Y}(x) = F_{\xi \vert \xi >0}(x) = P(\xi < x \vert \xi > 0) = \frac{F_{X}(x) - P(X = 0)}{P(X >0)} = \frac{F_{X}(x) - p}{1 - p} \quad x>0$$
где $F_{X}(x)$ - распределени ущерба, реализовался он или нет.

Выразим функцию распределения ущерба:
$$F_{X}(x) = P(X<x) = P(X < x \vert X = 0) \cdot P(X = 0) + P(X <x \vert X >0) \cdot P(X>0) = $$
$$ = p + F_{\xi \vert \xi >0}(x) \cdot (1-p) = p + F_{Y}(x) \cdot (1-p)$$
где $p = \lim\limits_{x \to 0+} F_{X}(x)$

Можно записать, как $X = I \cdot Y$

Так же можно найти плотность при $x>0$:
$$f_{X}(x) = (1-p) \cdot f_{Y}(x)$$
$$p + (1-p)\int\limits_0^{\infty}f_{Y}(x)dx = 1$$

Вычислим распредление случайной величины $\xi = I \cdot Y$:
$$\mathbb{E}\xi = \mathbb{E}Y \cdot P(I = 1) = (1-p)\cdot \mathbb{E}Y = (1-p) \mu$$
$$\mathbb{E}\xi^2 = (1-p)EY^2 = (1-p)(DY + (EY)^2) = (1-p)(\sigma^2 + \mu^2)$$
$$\mathbb{D}\xi = (1-p)DY + (1-p)(EY)^2 - (1-p)^2(EY)^2) = (1-p)\mathbb{D}(Y) + (1-p)p(\mathbb{E}(Y))^2 =$$
$$= EIDY + DI(EY)^2 = (1-p)\sigma^2 + p(1-p)\mu^2$$

Найдем через условные математические ожидания:
$$E\xi = E(IY)=E(E(Y \vert I)) = E(\mu I) = (1-p)\mu$$
$$D\xi = D(IY) = E(D(Y \vert I)) + D(E(Y \vert I)) = E(\sigma^2I) +D(\mu I) = (1-p)\sigma^2 + p(1-p)\mu^2$$
\subsection{Модель индивидуального риска, функция распределения и характеристики суммарного ущерба. Преимущества и недостатки модели индивидуального риска. Подходы к оценке модели.}

Совокупный ущерб - сумма случайных величин индивидуальных ущербов:
$$S_{\operatorname{ind}} = \sum\limits_{j=1}^n \xi_j$$
где $n$ - объём портфеля. Часть слагаемых соответствующих тем рискам, по которым не было ущерба, равна нулю.

Распределение сумм случайных величин может осуществляться с помощью:
\begin{itemize}
	\item свёртки
	\item производящие функции моментов
\end{itemize}

Пусть $\xi, \eta$ - непрерывные случайные величины, тогда по формуле свёртки:
$$f_{\eta + \xi} = \int\limits_{-\infty}^{\infty} f_{\eta}(x)f_{\xi}(z-t)dt$$

Теперь найдем функцию распределения суммы двух непрерывных случайных величин:
$$F_{\eta + \xi} (s)= P(\eta + \xi \leqslant s) = P(\eta + \xi \leqslant s)$$

Для двух дискретных неотрицательных случайных величин мы можем воспользоваться формулой полной вероятности и записать в виде:
$$F_{\eta + \xi}(s) = \sum\limits_{\text{по всем } y \leqslant s} P(\eta + \xi \leqslant s \vert \eta = y)  \cdot P(\eta = y) = \sum\limits_{\text{по всем } y \leqslant s} P(\xi \leqslant s - y \vert \eta = y) \cdot P(\eta = y)$$

Если $\xi$ и $\eta$ независимы, то сумма можем быть переписана:
$$F_{\xi + \eta}(s) =  \sum\limits_{\text{по всем } y \leqslant s} F_{\eta}(s-y)f_{Y}(y)$$
$$f_{\xi + \eta}(s) =  \sum\limits_{\text{по всем } y \leqslant s} f_{\eta}(s-y)f_{Y}(y)$$

Для непрерывных неотрицательных случайных величин формулы имеют вид:
$$F_{\xi + \eta}(s) = \int\limits_0^s P(\xi \leqslant s-y \vert \eta = y)f_{\eta}(y)dy$$
$$F_{\xi + \eta}(s) = \int\limits_0^s F_{\xi}(s-y)f_{\eta}(y)dy$$
$$f_{\xi + \eta}(s) = \int\limits_0^s f_{\xi}(s-y)f_{\eta}(y)dy$$

Обозначение свертки для двух функций распределения $F_{\xi}(x)$ и $F_{\eta}(x)$ - $F_{\xi} * F_{\eta}$.

Для определения распределения суммы более чем двух случайных величин можем использовать итерации процесса свёркти. Для $S = \xi_1 + \ldots + \xi_n$, где $\xi_i$ - независимые случайные величины, $F_i$ обозначает функцию распределения случайной величины $\xi_i$, а $F^{(k)}$ - функция распределения $\xi_1 + \ldots + \xi_k$, мы получим:
$$F^{(2)} = F_2 * F^{(1)} = F_2 * F_1$$
$$F^{(3)} = F_3 * F^{(2)}$$
$$F^{(4)} = F_4 * F^{(3)}$$
$$f^{(2)} = \int\limits_0^x f_1(x-y)f_2(y)dy$$
$$f_{\xi_1 + \xi_2 + \xi_3} = f^{(3)}(x) = \int\limits_0^x f^{(2)}(x-y)f_3(y)dy$$

Достоинством свёртки является получение точного распределения. Недостаток - больший объём вычислений.

Воспользуемся производящими функциями моментов, которая для случайной величины $\xi$ определяется соотношением:
$$\psi_{\xi}(t) = \mathbb{E}e^{t\xi}$$
$$\psi_{\xi_1 + \xi_2} = E\left(e^{t(\xi_1 + \xi_2)}\right) = E\left(e^{t\xi_1}e^{t\xi_2}\right)$$

В случае независимости:
$$\psi_{\xi_1 + \xi_2} = E\left(e^{t(\xi_1 + \xi_2)}\right) = E\left(e^{t\xi_1}e^{t\xi_2}\right) = E(e^{t\xi_1}) \cdot E(e^{t\xi_2}) = \psi_{\xi_1} \cdot \psi_{\xi_2} $$
$$\psi_{\xi_1 + \xi_2} =  \psi_{\xi_1} \cdot \psi_{\xi_2}$$

Это свойство распространяется на сумму любого детерменировнного числа независимых случайных величин:
$$\psi_{S_{\operatorname{ind}}}(t) = \prod\limits_{k=1}^n \psi_{\xi_k}(t)$$

Ограничением данного подхода является то, что производящие функции моментов определеные не для всех типов распределений. Если одинаково распределеные, то модель индивидуального риск будет относиться к тому же классу, что и распределение каждого индивидуального ущерба. Если $n \to \infty$, то можно воспользовться асиметотическими свойствами.

Посчитаем математическое ожидание и дисперсию:
$$E(S_{ind}) = \sum\limits_{j=1}^n E(\xi_j)$$
$$D(S_{ind}) = \sum\limits_{j=1}^n D(\xi_j) + 2 \sum\limits_{j=1}^{n-1}\sum\limits_{k=j+1}^n \operatorname{Cov}(\xi_j, \xi_k)$$
Для независимых случайных величин:
$$E(S_{ind}) = n(1-p)\mu$$
$$D(S_{ind}) = n(1-p)\sigma^2 + np(1-p)\mu^2$$

Для неоднорожных портфелей (то есть для портфелей, имеющих различное распределение случайных величин):
$$E(\xi_j) = (1-p_j)\mu_j \quad D(\xi_j) = (1-p_j) \cdot \sigma_j^2 + p_j (1-p_j)\mu_j^2; \quad  j = 1, \ldots, n$$
$$E(S_{ind}) = \sum\limits_{j=1}^n (1-p_j)\mu_j$$
$$D_{S_{ind}} = \sum\limits_{j=1}^n (1-p_j)\sigma_j^2 + \sum\limits_{j=1}^n p_j(1-p_j)\mu_j^2$$
для независимых случайных величин.

\subsection{Модель коллективного риска, функция распределения и характеристики суммарного ущерба. Преимущества и недостатки модели коллективного риска. Подходы к оценке модели.}

В отличие от модели индивидуального риска, где неопределенность, связанная
с размером ущерба, отражалась в специфическом виде их распределений (со скачком в нуле), в модели коллективного риска неопределенность, связанная с числом
случаев возникновения ущерба, отделяется от неопределенности, вызванной размером ущерба. При этом совокупный ущерб моделируется как сумма случайного числа
случайных величин:
$$S_{coll} = \sum\limits_k^N Y_k$$
где $N$ - случайная величина числа случае возникновения неблагоприятных событий, $Y_k$ - случайная величина числа размера ущерба (усеченное распределение $Y_k >0$).

Таким образом, в модели коллективного риска четко выделяются два типа
неопределенностей, связанных с количеством случаев возникновения ущерба и размером ущерба. Для этой модели обычно применяют не закон больших чисел и другие
асимптотические результаты (хотя это тоже возможно), а методы анализа случайных процессов. Фактически Scoll можно интерпретировать как значение случайного
процесса в случайный момент времени.

Усложнение применяемого математического аппарата является очевидным
недостатком указанной модели. К преимуществам следует отнести возможность разделения анализа числа неблагоприятных событий и размера ущерба, что служит реализации задач риск-менеджера в свете специфических ограничений информационного обеспечения.

Событие $\{S_{coll} < s\}$ - объединение непересекающихся событий:
$$\cup_{n=0}^{\infty} \{\sum\limits_{k=1}^n Y_k < s, N = n\}$$

Поэтому:
$$P(S_{coll} < s) = \sum\limits_{i=1}^n P(\sum\limits_{k=1}^n Y_k < s, N = n) = \sum\limits_{i=1}^n P(\sum\limits_{k=1}^n Y_k < s \vert N = n) \cdot P(N = n)$$
$$F_{S_{coll}}(s) = \sum\limits_{n} P(N=n)F_{Y_k}^{*n}(s)$$
где $F_{Y_k}^{*n}(s)$ - $n$-кратная свертка случайной величны $Y_k$ при этом $F^{*1}(s) = F(s)$ и $F^{*0}(s) = 1, x \geqslant 1$.

Для упрощения расчетов предполагают, что случайные числа одинаково распределенные, независимые, ковариации равны нулю между любыми случайными величинами. Ограничивает применить - упрощает математические методы.

Посчитаем математическое ожидание и дисперсию:
$$E(S_{coll}) = E(\sum\limits_{k=1}^n Y_k) = E(E(\sum\limits_{k=1}^n Y_k \vert N)) = \sum\limits_n \left(\sum\limits_{k=1}^n Y_k\right) \cdot P(N = n) = $$
$$=\sum\limits_n n E(Y_k)P(N=n) = E(Y_k) \cdot E(N)$$
$$D(S_{coll}) = (E(Y_k))^2D(N) + D(Y_k)E(N)$$

\subsection{Производящая функция коллективного риска}

Производящая функция коллективного риска:
$$\psi_{S_{coll}}(t) = E(e^{tS_{coll}}) = E(E(e^{tS_{coll}}\vert N)) = E((\psi_{Y_k}(t))^N) = E(\exp (N \ln \psi_{Y_k}(t))) = $$
$$ = \psi_N(\ln (\psi_{Y_k}(t))) = G_N(\psi_{T_k}(t))$$

\subsection{Отличия моделей коллективного и индивидуального риска}

Хотя предпосылки обоих подходов несколько отличаются, на интуитивном
уровне различия сводятся к специфике учета в модели рисков, по которым ущерб не
возник: в модели индивидуального риска они «отвечают» за скачок функции распределения, а в модели коллективного риска их игнорирование «оплачивается» рандомизацией числа неблагоприятных событий. Это делает весьма вероятным близкое
соответствие результатов моделирования совокупного ущерба обоими способами.

\subsection{Примеры считающих распределений. Смеси}

$$\xi = p \cdot \eta + (1-p)(\eta_1 + \eta_2)$$
$$F_{\xi}(x) = p(\xi <x) = P(\eta < x) \cdot p + P(\eta_1 + \eta_2 < x)(1-p) = F_{\eta}p + F_{\eta_1 + \eta_2} \cdot (1-p)$$

\newpage

\subsection{Домашнее задание. Моделирование рисков (I)}

3. Ущерб реализуется с вероятностью $0.2$, при этом в случае реализации величина ущерба определяется случайной величиной с функцией распределения $F_{\xi}(x) = \sqrt{\frac{x}{4}}, x \in [0;4]$. Вероятность более чем однократной реализации ущерба считается принебрежимо малой. Найти:

\begin{itemize}
	\item безусловную функцию распределения ущерба.

	\textbf{Решение:} Ущерб реализуется с вероятностью $0.2 \Rightarrow P(\xi > 0) = 0.2$. Отсутствие ущерба: $p = 0.8$. Если точно известно, что ущерб был, то его функция распределения:
	$$F_{Y}(x) = F_{\xi \vert \xi >0}(x) = P(\xi < x \vert \xi > 0) = \frac{F_{X}(x) - P(X = 0)}{P(X >0)} = \frac{F_{X}(x) - p}{1 - p} \quad x>0$$
	где $F_{X}(x)$ - распределение ущерба, реализовался он или нет. Тогда безусловная функция распределения ущерба:
	$$F_{X}(x) = F_{\xi \vert \xi >0} \cdot (1-p) + p = \sqrt{\frac{x}{4}} \cdot 0.2 + 0.8, x \in [0;4]$$
	\item математическое ожидание ущерба.

	\textbf{Решение:} найдем плотности и по формуле математического ожидания высчитаем его
	$$f_{X}(x) = (1-p)f_{Y}(x)$$
	$$p + (1-p)\int\limits_0^{\infty}f_{Y}(x)dx = 1$$
	$$f_{X}(x) = \frac{\partial f_{X}(x)}{\partial x} = \frac{0.05}{\sqrt{x}}$$
	$$f_{Y}(x) = \frac{f_{X}(x)}{1-p} = \frac{1}{4\sqrt{x}}$$
	$$E(X) = \int\limits_0^4 x \cdot f_{X}(x) dx = \int\limits_0^4 0.05\sqrt{x} = 0.26$$
	\item дисперсию ущерба
	$$D(X) = \int\limits_0^4 (x - E(X))^2 f_{X}(x) dx = 0.512$$
	\item математическое ожидание ущерба если известно, что он точно реализовался:
	$$E(X) = (1-p)EY \Rightarrow E(Y) = \frac{E(X)}{1-p} = \frac{0.26}{0.2} =1.33$$
	\item $\operatorname{VaR}_{95\%}$ ущерба:
	$$F_{X}(x_{\alpha}) = \alpha \Rightarrow  \sqrt{\frac{x}{4}} \cdot 0.2 + 0.8 = 0.95$$
	$$x_{\alpha}  = 2.25$$
	\item математическое ожидание ущерба, если точно известно, что оно больше двух:
	$$E(X \vert X > 2) = \frac{1}{\int\limits_2^{4} f_{X}(x) dx} \cdot \int\limits_2^{4} xf_{X}(x)dx = 2.94$$

\end{itemize}

4. Величина ущерба определяется случайной величиной с функцией распределения равной $0$, если $x \leqslant 0$ и $1 - (10(x+1)^4)^{-1}$. Найти:

\begin{itemize}
	\item вероятность отсутствия ущерба:
	$$p = 1 - (10(0+1)^4)^{-1} = 0.9 $$
	\item функция распределения ущерба, если известно, что он реализовался:
	$$F_{\xi \vert \xi >0} = \frac{F_{X}(x) - p}{1-p} = - 9  - \frac{1}{(x+1)^4}$$
	\item математическое ожидание ущерба, если известно, что ущерб реализовался:
	$$E(X) = \int\limits_0^{\infty} x \cdot f_{X}(x) = -\frac{1}{30}$$
	$$E(Y) = \frac{E(X)}{1-p} = -\frac{1}{30} \cdot 10 = -\frac{1}{3}$$
	\item дисперсия ущерба:
	$$D(X) = \int\limits_0^{\infty} (x - E(X))^2 f_{X}(x) dx = -0.03$$
\end{itemize}

6. Компания подвержена двум рискам, первый из которых реализуется с вероятностью $0.2$, а второй - $0.1$, при этом первый риск в случае реализации приводит к ущербу, определяемому экспоненциальной случайной величиной с параметром $1$, а второй  - с параметром $2$. Найти:

\begin{itemize}
	\item функцию распределения суммарного ущерба компании:
	$$F_{\xi}(x) = \frac{p_1}{p_1+p_2} \cdot (1  - e^{-x}) + \frac{p_2}{p_1 + p_2}(1 - e^{-2x})$$
	\item математическое ожидание ущерба:
	$$f_{\xi}(x) =\frac{p_1}{p_1+p_2}  \cdot e^{-x} + 2e^{-2x}\frac{p_2}{p_1+p_2} $$
	$$E(\xi) =  \int\limits_{0}^{\infty} x \cdot f_{\xi}(x) dx = 0.833333$$
	\item дисперсию ущерба:
	$$D(\xi) = \int\limits_0^{\infty} (x - E(\xi))^2 f_{\xi}(x) dx  = 0.805556$$
\end{itemize}

8. Пусть количество реализовавшихся ущербов может принимать значения $0, 1, 2$ с вероятностями $0.7$, $0.2$, $0.1$, при этом величина каждого ущерба определяется случайной величиной с распределением $\xi \sim N(10,1)$.

\begin{itemize}
	\item Найти вероятность того, что итоговый суммарного ущерба окажется меньше $11$. Воспользуемся свойством, что есть $\xi \sim N_{\mu_1, \sigma_1^2}$ и $\eta \sim N_{\mu_2, \sigma_2^2}$ независимы, то $\xi + \eta \sim N_{\mu_1 + \mu_2, \sigma_1^2 + \sigma_2^2}$
	$$F_{S} = P(S < s) = \sum\limits_{n=0}^2 P(N = n)F_{Y_k}^{*n}(s) = $$
	$$ = P(N=0) \cdot F^{*0}(s) +  P(N=1) \cdot F^{*1}(s) +  P(N=2) \cdot F^{*2}(s) = $$
	$$ = 0.7 * 1 + 0.2 \cdot \Phi_{\xi}(11) + 0.1 \cdot  F^{*2}(2)$$
	\item найти математическое ожидание, дисперсию и 99 процентный квантиль
	$$E(S_{coll}) = E(\sum\limits_{k=1}^n Y_k) = E(E(\sum\limits_{k=1}^n Y_k \vert N)) = \sum\limits_n \left(\sum\limits_{k=1}^n Y_k\right) \cdot P(N = n) = $$
$$=\sum\limits_n n E(Y_k)P(N=n) = E(Y_k) \cdot E(N) = 0.4 \cdot 10 = 4$$
$$D(S_{coll}) = (E(Y_k))^2D(N) + D(Y_k)E(N) = 100 \cdot 0.44 + 1 \cdot 0.4 = 45.04$$
\end{itemize}

10. Пусть количество реализовавшихся ущербов описывается Пуассоновской случайной величной с параметром $3$, а размер ущерба детерменирован и равен $1000$. 
\begin{itemize}
	\item Опишите закон распредлеения суммарного ущерба:
	$$F_{X}(x) =\begin{cases}
		0, x \leqslant 0 \\
		1000 \cdot e^{-\lambda} \cdot \sum\limits_{j=0}^k \frac{\lambda^j}{j!}, k < x \leqslant k+1, k = 0, 1, 2
	\end{cases}$$

\end{itemize}


\subsection{Считающие распределения 05.11}

Можно рассматривать случайные величины, параметры которых - тоже случайные величины. Например $\xi \sim N(\mu, \sigma^2)$, где \(\sigma \{1,10\} \mapsto \{0.9, 0.1\}\).

Например, отрицательное биномиальное распределение может быть получено из пуассоновского рандомизацией параметра $\lambda$. Надо бы разобрать несколько канонических распределений. А будем обсуждать задачи одномерного моделирования.

\subsection{Теория экстремальных значений}

Одна из важных задач процессов управления рисками - контроль ситуаций, связанных с экстремально большими ущербами. С точки
зрения количественного риск-менеджмента речь идет прежде всего об анализе пра-
вого хвоста распределений ущерба, как раз и описывающего вероятности серьезных
потерь. Такие меры риска, как рисковый капитал и условный рисковый капитал,
концентрируются именно на данных особенностях случайных величин.

Мы обсуждали непараметрический и параметрический (подгонка теоретического расределения).  Минусом непараметрического оценивания является большая чувствительность к выборочным данным. Изменение одного числа в верхней части выборки может привести к сереьезным изменениям оценки исследуемого параметра.

Параметрическое оценивание - на параметры модели влияют все элементы выборки - старшие квантили могут недооцениваться - характерно для распределения с тяжёлыми хвостами. Проблему быстрого убвыания хвоста нормального распределения - используем стбюдента и смеси нормального распределения.

Другая проблема - риски, характеризующиеся малой частотой реализации, но большими убытками. Дамба, наводнение.

\textbf{Первый метод:} - максимальный ушерб в течение периода фиксированной длины на основе выборки максимальных ущербов.

\textbf{Второй метод:} анализ ущербов, превышающих некоторый заранее выбранный поолог, на основе выборки из прошлых ущербов, превысивших порог.

\subsubsection{Метод анализа распределения максимального ущерба за период}

Пусть наблюдаемые данные $X_i$ представляют выборку из ущербов за день, то максимальный дневной ущерб за неделю будет описываться случайной величиной $Y=\max\{X_1, \ldots, X_7\}$.

Если предпологать $X_i$ независимыми одинаково распределенными случайными величинами, то функция распределения $Y$и $X_i$:
$$F_{Y}(x) = F_{\underset{i}{\max} X_i}(x) = P\{\underset{i}{\max} \{X_i\} \leqslant x\} = (F_{X_i}(x))^n$$

\begin{theorem}
	Теорема Фишера-Типпета
\end{theorem}

Вывод теории экстремальных значений: если удается подобрать такие последовательности $b_n, a_n$, что \(\left(F_{X_i}\left(\frac{x-b_n}{a_n}\right)\right)^n\) - невырождено при $n \to \infty$ (не стремится к $0$ или $1$), то для независимых одинаково распределенных случайных величин выполняется соотношение:
$$\left(F_{X_i}\left(\frac{x-b_n}{a_n}\right)\right)^n \to H_{\beta}(x)$$
где $H_{\beta}(x)$ - обобщенное распределение экстремальных значений (\textsc{Generalized Extreme Value Distribution, GEV Distribution}). 

Функция обобщённого распределения экстремальных значений выглядит следующим образом:
$$H_{\beta}(x) = \begin{cases}
	e^{-(1+\beta x)^{-\frac{1}{\beta}}}, \beta \neq 0 \\
	e^{-e^{-x}}, \beta = 0
\end{cases}$$
где $1 + \beta x < 0$. Без нормировки будет стремиться либо к $0$ либо к $1$.

Если вместо рассмотрения выборки разбить её на равные периоды длительности $n$, выбрть из каждого максимальный ущерб и составить новую выборку из выбранных значений, то новая выборка будет хорошо описываться обобщенным распределением экстремальных значений.

GEV зависит от параметра $\beta$. При $\beta > 0$ - распределение Фреше, $\xi < 0$ - распределение Вейбулла, $\xi =0$ - распределение Гумбеля.

\textbf{Пример:} пусть случайные величины $X \sim Exp(1), F_{\xi}(x) = 1 - e^{-x}, x>0$. В качестве последовательностей нормирующих констант возьмем $b_n = -\ln n, a_n = 1$. $Y = \max(X_1, \ldots, X_n)$:
$$F_{\xi}^n(x) = P\left(\frac{Y-b_n}{a_n} < x\right) = P(Y < a_nx + b_n) = $$
$$ = P(X < x - \ln n)^n = (1 - e^{-x+\ln n})^n = \left(1 - \frac{e^{-x}}{n}\right)^n \to e^{-e^{-x}}$$

Если теорема Фишера-Типпета верна для $F(x)$, то $F(x)$ принадлежит к \textbf{максимальной области притяжения} $G_{\xi}(x)$ - MDA - maximum domain of attraction. Наиболее тяжёлые хвосты характерны для распределения Фреше ($\beta > 0$): обратное гамма-распределение, $t$-распределение, логгамма распределение, распределение Бёрра, распределение Парето, Коши.
$$\lim\limits_{x \to \infty} \frac{L(ax)}{L(x)} = 1, a>0$$

Если $F(x) \in MDA(G_{\xi}), \beta >0$, то $1 - F(x) = x^{-\frac{1}{\beta}}L(x)$, где $L(x)$ - некоторая медленно меняющаяся функция. $\frac{1}{\beta}$ - звостовой индекс распределения.

Распределение Гумбеля $\beta=0$: нормальные распределения, логнормальные, гамма, хи-квадрат, гиперболические - имеют конечные моменты всех порядков.

Распределения из MDA Вейбулла $\beta <0$ - наименьший интерес с точки зрения анализа рисков. $GEV$-непрерывное по $\beta$. 

Недостатки максимального ущерба: потеря большого числа наблюдений выборки, предположение независимости и одинаковой расределенности, что часто не выполняется. Важность: анализ дамбы. Зная максимальный за год - можно за $100$ лет - произведение функции распределений в 100 степени.

\subsection{Распределение максимума из случайного числа случайных величин}

Предложенный выше метод подразумевает анализ максимума из фиксирован-
ного числа случайных величин, однако в некоторых приложениях интерес пред-
ставляет анализ максимума из случайного числа случайных величин. Использова-
ние предельной теоремы здесь не представляется корректным. Для изучения распределения подобной случайной величины удобно использовать производящую функцию вероятснотей, $\varphi_X(z) = E(z^X)$. $Y_N = \max(X_1,\ldots, X_N)$, где $N$ определяет число реализовавшихся ущербов. В этом случае в предположении о независимости $N$ и $X_i$:
$$F_{Y_N}(x) = P(Y_N < x) = \sum\limits_{i=0}^{\infty}P(Y_N < x \vert N = i)P(N=i) = $$
$$ = \sum\limits_{i=0}^{\infty}P(\max(X_1, \ldots, X_i) <x) P(N=i) = $$
$$ = \sum\limits_{i=0}^{\infty}P(X<x)^iP(N=i) = \varphi_N(F_{X}(x))$$

\subsection{Метод анализа распределения превышения заданного порога}

Метод придлагает выбрать некоторый достаточно большой порог и будем исследовать распределение лишь ущербов, превышающих.

Пусть $d$ - фискированный порог, $Y = X - d$, при условии $X \geqslant d$. Случайная величина определяет на сколько ущерб превысил данный уровень $d$. 

Функция распределения $Y$:
$$F_{Y}(x) = P(X-d<x \vert X \geqslant d) = \frac{P(d\leqslant X < d+x)}{P(X \geqslant d)} = \frac{F_{X}(d+x) - F_{X}(x)}{1-F_{X}(x)}$$
где $0 \leqslant x \leqslant x_F = \sup\{t:F(t) < 1\}$

Математическое ожидание $Y$ - \textbf{функция среднего превышения}:
$$e(d) = E(Y) = E(X-d \vert X \geqslant d)$$
$$1 - F_{X}(x) = (1-F_{X}(d))(1-F_{Y}(x-d))$$
и оценка квантилей распределения $X$ сводится к задаче оценки величины $F_{X}(d)$ и анализу функции распределения $Y$.

\begin{theorem}
	Теорема Balkema, Pickands, de Haan. Связь превышения порога с проблемой максимума:
\end{theorem}

Если (и только если) $X$ принадлежит к максимальной обласит притяжения одного из обобщенных распределений экстремальных значений с параметром $\beta$, то можно найти такую функцию: $k(d)$:
$$\lim\limits_{d \to x_F} \sup\limits_{0 \leqslant x \leqslant x_F-} \vert F_{Y}(x) - W_{\beta, k(d)}(x) \vert = 0$$
где $W$ - Обобщенное распределение Парето (generalized Pareto Distribution, GPD). Чья функция распределения:
$$W_{\beta, \alpha} = \begin{cases}
	1 - \left(1+\frac{\beta x}{\alpha}\right)^{-\frac{1}{\beta}}, \beta \neq 0 \\
	1 - e^{-\frac{x}{\alpha}}, \beta  = 0
\end{cases}$$

\begin{itemize}
	\item $\beta > 0$ - преобразование распределения Парето
	\item $\beta < 0$ - бета распределение
	\item $\beta = 0$ - экспоненциальное распределение.
\end{itemize}

$GEV$ непрерывны по $\beta$.
$$GEV = 1 + \ln GPT$$

Если $X$ экспоненциально, то и $Y$.
$$E(X) = \frac{\alpha}{1 - \beta}$$
$$CVaR_{\alpha}(X) =\frac{Var_{\alpha}(X)}{1-\beta} + \frac{\delta}{1-\beta}$$

Метод порога - предпочтительнее - использует больше информации, не игнорировать большие ущербы, произошедшие на протяжении небольшого отрезка времени. Минусы: независимость и одинаково распределенность.

Экспоненциальное притягивается к Гумбелю.



\end{document}