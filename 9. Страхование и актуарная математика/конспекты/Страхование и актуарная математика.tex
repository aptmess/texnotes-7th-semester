\RequirePackage{ifluatex}
\let\ifluatex\relax

\documentclass[aps,%
12pt,%
final,%
oneside,
onecolumn,%
musixtex, %
superscriptaddress,%
centertags]{article} %% 
\topmargin=-40pt
\textheight=650pt
\usepackage[english,russian]{babel}
\usepackage[utf8]{inputenc}
%всякие настройки по желанию%
\usepackage[colorlinks=true,linkcolor=black,unicode=true]{hyperref}
\usepackage{euscript}
\usepackage{supertabular}
\usepackage[pdftex]{graphicx}
\usepackage{amsthm,amssymb, amsmath}
\usepackage{textcomp}
\usepackage[noend]{algorithmic}
\usepackage[ruled]{algorithm}
\usepackage{lipsum}
\usepackage{indentfirst}
\usepackage{babel}
\usepackage{pgfplots}
\usepackage{setspace}
\linespread{1.2}
\pgfplotsset{compat=1.9}
\selectlanguage{russian}
\pgfplotsset{model/.style = {blue, samples = 100}}
\pgfplotsset{experiment/.style = {red}}
\theoremstyle{plain}
\binoppenalty=10000
\newtheorem{theorem}{Теорема}[section] %
\setlength{\parindent}{2.4em}
\setlength{\parskip}{0.1em}
\theoremstyle{definition}
\newtheorem{definition}{Определение}[subsection]
\theoremstyle{remark}
\newtheorem{remark}{Замечание}[section]

\newtheorem{corollary}{Следствие}
\newtheorem{proposition}{Proposition}
\newtheorem{example}{Пример}
\renewcommand*{\proofname}{Proof}

\newtheorem{lemma}{Лемма}[section]

\graphicspath{ {./image/} }
\usepackage{xcolor}
\usepackage{hyperref}


\begin{document}

\begin{titlepage} 
\begin{center}
\textbf{}\\[10.0cm]
\textbf{\LARGE Страхование и актуарная математика}\\[0.5cm]
\textbf{\Large Александр Широков ПМ-1701} \\[0.2cm]


\begin{center} \large
{Преподаватель:} \\[0.5cm]
\textsc {Радионов Андрей Владимирович}\\
\end{center}

\vfill 



{\large {Санкт-Петербург}} \par
{\large {2020 г., 7 семестр}}
\end{center} 
\end{titlepage}

% Table of contents
\begin{thebibliography}{3}
  \bibitem{A}
\end{thebibliography}
\tableofcontents
\newpage

\section{01.09.2020}

\subsection{О чём предмет}

Страхование вклада - если у банка возникают проблемы, то нам возвращают деньги банк в некоторой границе. Страхование - выход США из великой депрессии. Застраховать в принципе можно все что угодно.

\section{03.09.2020}

В ситуациях неопределенностей человек принимает решение не на основании математического ожидания $E\xi$, а на основании математического ожидания некоторой функции полезности $Eu(\xi)$, где $u$ - некая функция полезности. За $w$ - обозначим капитал, а за $a$ - плата за риск, $\xi$ - потенциальные убытки. Тогда ситуация будет описываться:
$$Eu(w-\xi) \quad u(w-a)$$

Пусть $W=100$ и случайая величина убытков принимает следующие значения: $0$ с вероятностью $0.9$,$1$ с вероятностью $0.05$,$20$ с вероятностью $0.05$. Функция полезности - $u(x) = \ln(x+1)$. Математическое ожидание убытка $E\xi = 0.55$. Приходит банк и говорит продать за $60$.

$w-\xi$ - начальное состояние, $w-a$ - возможное состояние, сравнение полезностей

Посчитаем:
$$Eu(w-\xi) = E \ln(100-\xi+1) = \ln (101-0+1) \cdot 0.9 + \ln (100-1+1) \cdot 0.05+ \ln (100-10+1) \cdot 0.05 = 4.60941$$
$$E \ln(100+1-0.55) = 4.60966$$

Есть $W, u(\xi), \xi, f_{\xi}(x)$:
$$E\left(u(W-\xi)\right) = \int\limits_{-\infty}^{\infty}u(W-x)f_{\xi}(x)dx$$

Вместо бесконечностей используются границы интегрирования.
$$\Delta u = \frac{\Delta W}{W} \qquad du = \frac{d W}{W} \qquad u = \ln W$$

\begin{definition}
	Пусть есть набор случайных величин $\xi$ и будем задавать предпочтение подобным образом $\xi \geq \eta$ - предпочтение нестрого отношение, если существуют какие-то пары, которые находятся в бинарном отношении.

	Мы будем говорить про свойства отношений:
	\begin{enumerate}
		\item Пиолнота: $\xi \geq \eta$ или $\eta \geq \xi$
		\item Транзитивность: $\xi \geq \eta, \eta \geq \varepsilon$ $\Rightarrow$ $\xi \geq \varepsilon$
		\item Из первого следует рефлексиновть
	\end{enumerate}

	Будем говорить, что данное бинарное отношение является отношением \textit{эквивалентности} - рефлексивно, транзитивно, симметрично.
\end{definition}
\begin{definition}
	Будем говорить, что $\xi \geq \eta$ и $\xi \not \sim \eta$ - отношение строго порядка
\end{definition}
\begin{definition}
	$V: \Xi \to R$ - функция $V$ сохраняет упорядочивание, если $\xi \geq \eta$, то:
	$$V(\xi) \geq V(\eta)$$
\end{definition}
\begin{definition}
	Пусть есть набор $\mathbb{A}_j$. $B \in A$ является полным по упорядочиванию, если для любых элементов $a,b \in A$ существует элемент $\exists c \in B$, что либо $a \geq c > b$ либо $a >c \geq b$.
\end{definition}
\begin{theorem}
	На $\Xi \leq V$ существует отношение, сохраняющее отношение, тогда и только тогда, когда в $\Xi$ существует счетно или конечное подмножество плотное по упорядочиванию.
\end{theorem}

Как построить функцию полезности? Построим отношение порядка на множестве товаров, строим кривые безразличия - классы эквивалентности (все элементы внутри эквивалентны между собой), они не пересекаются.

Построим прямую, единичный вектор (бисскетриса). На что нужно умножить единичный вектор, чтобы попасть в точку пересечения, и высчитываем функцию полезности.

$$V^*(\xi) = Eu(\xi) = \sum\limits u(x_i)p_i = V(x_1)p_1+V(x_2)p_2$$
$$\xi: x_1 \mapsto p_1,  x_2 \mapsto p_2$$
Мы можем выбрать функцию полезности таким образом

\section{04.09.2020}

Пусть случайная величина принимает значения $x \mapsto p$ и $y \mapsto 1-p$. Введем обозначение для такой случайной величины:
$$(x,y)_p$$

Наложим некоторые ограничения:
\begin{itemize}
	\item Тогда индивид индеферентен: $(x,y)_1 \sim x$.

	Например: пусть $\xi$ равномерно распределен на отрезке $[0,1]$: $\xi \sim \mathbb{U}[0,1]$.
	\item $(x,y)_p \sim (y,x)_{1-p}$
	\item $((x,y)_p,y)_q \sim (x,y)_{pq}$

	Пример: $(1,(2,3)_{\frac{1}{2}})_{\frac{1}{2}}$ для игры $1 \mapsto \frac{1}{2},2 \mapsto \frac{1}{4},3 \mapsto \frac{1}{4}$
\end{itemize}

Ограничения:
\begin{itemize}
	\item $\{p \in [0,1]: (x,y)_p \geq z\}$ - замкнутное множество
	\item $\{p \in [0,1]: z \geq (x,y)_p \}$ - замкнутное множество $\forall x,y,z \in \Xi$
	\item $x \sim y: (x,z)_p \sim (y,z)_p$
	\item $\exists w,b: \forall \xi \in \Xi$ выполняется, что $w \geq x \geq b$
	\item $(b,w)_p \geq (b,w)_q \Leftrightarrow p > q$
\end{itemize}

\begin{theorem}
	Если $\Xi$ и на нём введено отношение предпочтения $\geq$, то найдётся такая функция $V$, что $$V((x,y)_p) = pV(x)+(1-p)V(y)$$
	$$Eu(\xi) = (x,y)_p$$
	$$V(\xi) = Eu(\xi)$$
	$$u(y): R \to R, V: \Xi \to R$$
\end{theorem}

Доказательство: Хэливэриан.

Рассмотрим некоторый функционал $V((x,y)_p)$ и применим к нему некоторое преобразование:
$$f\left(V((x,y)_p)\right) = f(V(x)\cdot p+V(y)\cdot (1-p)) = f(V(x))\cdot p + f(V(y))\cdot (1-p)$$

и это линейная функция - линейное преобразование.

Пример: $u(x) = \ln (x+1)$, $\xi, f_{\xi}(y), Eu(\xi) = \int\limits u(x)f_{\xi}(x)dx$

$u_1(x) = a\ln(x+1) + b, a>0$
$$Eu_1(\xi) =  \int\limits  a\ln(x+1) + bf_{\xi}(x)dx = a\int ln(x+1)f(x)dx+b\int\limits f(x)dx = a\int\limits u(x)f_{\xi}(x)dx + b$$
следовательно функционал является единственным с точностью до линейного преобразования.




\end{document}