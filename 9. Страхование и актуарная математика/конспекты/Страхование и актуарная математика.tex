\documentclass[%
12pt, %
final, % 
oneside, % 
onecolumn, %  
centertags]{article} % относится к классу article и размер шрифта 12 пунктовб, {article: статья, report: отчеты и диссертации, book: книга, letter: письмо}

% ------ page construction 

\topmargin= -30pt % насколько сверху будет страница
\textheight= 650pt

% ------ Пакеты расширения

\usepackage[utf8]{inputenc} % задает кодировку, utf-8 кодировка, включающая в себя знаки почти всех языков мира
\usepackage[english, russian]{babel} % подключает необходимые языки, основным языком является английский
\selectlanguage{russian} % настройки будут на английском, но писать будет на русском

\usepackage{euscript}
\usepackage{supertabular}

\usepackage[colorlinks=true,linkcolor=black,unicode=true,urlcolor = blue]{hyperref} %hypered
\usepackage[pdftex]{graphicx} % для графики

\usepackage{amsthm, amssymb, amsmath, amsfonts} % математический пакет, математические шрифты
\usepackage{textcomp}
\usepackage[noend]{algorithmic}
\usepackage[ruled]{algorithm}
\usepackage{lipsum}
\usepackage{indentfirst}
\usepackage{babel}
\usepackage{pgfplots}
\usepackage{setspace}
\usepackage{xcolor}
\usepackage{hyperref}

\linespread{1.2} 
\setlength{\parindent}{2.4em}
\setlength{\parskip}{0.1em}

\pgfplotsset{compat=1.9}
\pgfplotsset{model/.style = {blue, samples = 100}} 
\pgfplotsset{experiment/.style = {red}}

\theoremstyle{plain}
\binoppenalty=10000

\newtheorem{theorem}{Теорема}[section] % theorem

\theoremstyle{definition}
\newtheorem{definition}{Определение}[subsection]

\theoremstyle{remark}
\newtheorem{remark}{Замечание}[section]

\newtheorem{corollary}{Следствие}

\newtheorem{proposition}{Proposition}

\newtheorem{example}{Пример}

\newtheorem{lemma}{Лемма}[section]

\renewcommand*{\proofname}{Proof}

\graphicspath{ {./image/} }


\begin{document}

\begin{titlepage} 
\begin{center}
\textbf{}\\[10.0cm]
\textbf{\LARGE Страхование и актуарная математика}\\[0.5cm]
\textbf{\Large Александр Широков ПМ-1701} \\[0.2cm]


\begin{center} \large
{Преподаватель:} \\[0.5cm]
\textsc {Радионов Андрей Владимирович}\\
\end{center}

\vfill 



{\large {Санкт-Петербург}} \par
{\large {2020 г., 7 семестр}}
\end{center} 
\end{titlepage}

% Table of contents
\begin{thebibliography}{3}
  \bibitem{A}
\end{thebibliography}
\tableofcontents
\newpage

\section{Конспекты лекций}

\subsection{Микроэкономические основы страхования}

\subsubsection{01.09.2020}

Страхование вклада - если у банка возникают проблемы, то нам возвращают деньги банк в некоторой границе. Страхование - выход США из великой депрессии. Застраховать в принципе можно все что угодно.

\subsubsection{03.09.2020}

В ситуациях неопределенностей человек принимает решение не на основании математического ожидания $E\xi$, а на основании математического ожидания некоторой функции полезности $Eu(\xi)$, где $u$ - некая функция полезности. За $w$ - обозначим капитал, а за $a$ - плата за риск, $\xi$ - потенциальные убытки. Тогда ситуация будет описываться:
$$Eu(w-\xi) \quad u(w-a)$$

Пусть $W=100$ и случайая величина убытков принимает следующие значения: $0$ с вероятностью $0.9$,$1$ с вероятностью $0.05$,$20$ с вероятностью $0.05$. Функция полезности - $u(x) = \ln(x+1)$. Математическое ожидание убытка $E\xi = 0.55$. Приходит банк и говорит продать за $60$.

$w-\xi$ - начальное состояние, $w-a$ - возможное состояние, сравнение полезностей

Посчитаем:
$$Eu(w-\xi) = E \ln(100-\xi+1) = \ln (101-0+1) \cdot 0.9 + \ln (100-1+1) \cdot 0.05+ \ln (100-10+1) \cdot 0.05 = 4.60941$$
$$E \ln(100+1-0.55) = 4.60966$$

Есть $W, u(\xi), \xi, f_{\xi}(x)$:
$$E\left(u(W-\xi)\right) = \int\limits_{-\infty}^{\infty}u(W-x)f_{\xi}(x)dx$$

Вместо бесконечностей используются границы интегрирования.
$$\Delta u = \frac{\Delta W}{W} \qquad du = \frac{d W}{W} \qquad u = \ln W$$

\begin{definition}
	Пусть есть набор случайных величин $\xi$ и будем задавать предпочтение подобным образом $\xi \geq \eta$ - предпочтение нестрого отношение, если существуют какие-то пары, которые находятся в бинарном отношении.

	Мы будем говорить про свойства отношений:
	\begin{enumerate}
		\item Пиолнота: $\xi \geq \eta$ или $\eta \geq \xi$
		\item Транзитивность: $\xi \geq \eta, \eta \geq \varepsilon$ $\Rightarrow$ $\xi \geq \varepsilon$
		\item Из первого следует рефлексиновть
	\end{enumerate}

	Будем говорить, что данное бинарное отношение является отношением \textit{эквивалентности} - рефлексивно, транзитивно, симметрично.
\end{definition}
\begin{definition}
	Будем говорить, что $\xi \geq \eta$ и $\xi \not \sim \eta$ - отношение строго порядка
\end{definition}
\begin{definition}
	$V: \Xi \to R$ - функция $V$ сохраняет упорядочивание, если $\xi \geq \eta$, то:
	$$V(\xi) \geq V(\eta)$$
\end{definition}
\begin{definition}
	Пусть есть набор $\mathbb{A}_j$. $B \in A$ является полным по упорядочиванию, если для любых элементов $a,b \in A$ существует элемент $\exists c \in B$, что либо $a \geq c > b$ либо $a >c \geq b$.
\end{definition}
\begin{theorem}
	На $\Xi \leq V$ существует отношение, сохраняющее отношение, тогда и только тогда, когда в $\Xi$ существует счетно или конечное подмножество плотное по упорядочиванию.
\end{theorem}

Как построить функцию полезности? Построим отношение порядка на множестве товаров, строим кривые безразличия - классы эквивалентности (все элементы внутри эквивалентны между собой), они не пересекаются.

Построим прямую, единичный вектор (бисскетриса). На что нужно умножить единичный вектор, чтобы попасть в точку пересечения, и высчитываем функцию полезности.

$$V^*(\xi) = Eu(\xi) = \sum\limits u(x_i)p_i = V(x_1)p_1+V(x_2)p_2$$
$$\xi: x_1 \mapsto p_1,  x_2 \mapsto p_2$$
Мы можем выбрать функцию полезности таким образом

\subsubsection{04.09.2020}

Пусть случайная величина принимает значения $x \mapsto p$ и $y \mapsto 1-p$. Введем обозначение для такой случайной величины:
$$(x,y)_p$$

Наложим некоторые ограничения:
\begin{itemize}
	\item Тогда индивид индеферентен: $(x,y)_1 \sim x$.

	Например: пусть $\xi$ равномерно распределен на отрезке $[0,1]$: $\xi \sim \mathbb{U}[0,1]$.
	\item $(x,y)_p \sim (y,x)_{1-p}$
	\item $((x,y)_p,y)_q \sim (x,y)_{pq}$

	Пример: $(1,(2,3)_{\frac{1}{2}})_{\frac{1}{2}}$ для игры $1 \mapsto \frac{1}{2},2 \mapsto \frac{1}{4},3 \mapsto \frac{1}{4}$
\end{itemize}

Ограничения:
\begin{itemize}
	\item $\{p \in [0,1]: (x,y)_p \geq z\}$ - замкнутное множество
	\item $\{p \in [0,1]: z \geq (x,y)_p \}$ - замкнутное множество $\forall x,y,z \in \Xi$
	\item $x \sim y: (x,z)_p \sim (y,z)_p$
	\item $\exists w,b: \forall \xi \in \Xi$ выполняется, что $w \geq x \geq b$
	\item $(b,w)_p \geq (b,w)_q \Leftrightarrow p > q$
\end{itemize}

\begin{theorem}
	Если $\Xi$ и на нём введено отношение предпочтения $\geq$, то найдётся такая функция $V$, что $$V((x,y)_p) = pV(x)+(1-p)V(y)$$
	$$Eu(\xi) = (x,y)_p$$
	$$V(\xi) = Eu(\xi)$$
	$$u(y): R \to R, V: \Xi \to R$$
\end{theorem}

Доказательство: Хэливэриан.

Рассмотрим некоторый функционал $V((x,y)_p)$ и применим к нему некоторое преобразование:
$$f\left(V((x,y)_p)\right) = f(V(x)\cdot p+V(y)\cdot (1-p)) = f(V(x))\cdot p + f(V(y))\cdot (1-p)$$

и это линейная функция - линейное преобразование.

Пример: $u(x) = \ln (x+1)$, $\xi, f_{\xi}(y), Eu(\xi) = \int\limits u(x)f_{\xi}(x)dx$

$u_1(x) = a\ln(x+1) + b, a>0$
$$Eu_1(\xi) =  \int\limits  a\ln(x+1) + bf_{\xi}(x)dx = a\int ln(x+1)f(x)dx+b\int\limits f(x)dx = a\int\limits u(x)f_{\xi}(x)dx + b$$
следовательно функционал является единственным с точностью до линейного преобразования.

\subsubsection{08.09.2020}

Напоминание:
$$V(\xi) = \sum\limits p_j V(x_j)$$
и при дискретных $\xi$:
$$V(x_j) = u(x_j)$$

Рассмотрим некоторые свойства, которыми должна обладать функция полезности:

\begin{enumerate}
	\item Функция начинается в нуле из-за монотонного преобразования
	\item Функция полезности $u(x)$ не убывает (возрастает):
	$$u'(x) > 0$$
	\item Функция $u(x)$ вогнутая. 
	$$u(\lambda x_1 +(1-\lambda)x_2) \geq \lambda u(x_1) + (1-\lambda)u(x_2)$$

	Заметим, что если функция полезности вогнутая, то находясь в ситуации неопределенности, индивид будет согласен заплатить, чем иметь состояние неопределенности. Человек хочет иметь детерменированный выигрыш, нежели при ситуации неопределенности - это происходит из-за вогнутости функции.
	\begin{itemize}
		\item есть функция вогнута, то говорят \textsc{risk aversion}
		\item если выпукла, то говорят \textsc{risk loving}
		\item если функция линейна, то \textsc{risk neutral}
	\end{itemize}

	Почитать \href{https://en.wikipedia.org/wiki/Risk_aversion}{здесь} можно.
	\item $$Eu(\xi) \leq u(E(\xi))$$
	
	Сравнивает полезность ситуации $u(w-a)$ - нет риска, чуть меньше денег, и есть риск и чуть больше денег - $Eu(w-\xi)$ и если больше, то он соглашается - страхование возможно для некоторого $a$ и человек готов заплатить. С помощью неравенства Йенсена:
	$$Eu(w-\xi)\leq u(E(w-\xi)) = u(w-E\xi)$$
	$$u(w-a) \geq Eu(w-\xi) \Leftrightarrow u(w-E\xi) = u(w-a)$$
	и следовательно мы сможем найти $a = E\xi$ из которого будет выполняться свойство.

	\begin{example}
		Возьмем экспоненциальную функцию полезности: $$u(x) = 1-e^{ux}$$. При желании для любой ограниченной функции можно подобрать лотерею так, в котороый можно подбирать математическое ожидание, чтобы человек всегда играл.

		В данном случае у нас ограниченная функция и ограниченное математическое ожидание.
	\end{example}
	\begin{example}
		Степенная функция полезности: $$u(x)=x^{\alpha}, \alpha < 1$$
	\end{example}
	\begin{example}
		Квадратичная функция полезности:
		$$u(x) = bx - cx^2: b,c >0, x < \frac{b}{2c}$$
	\end{example}
\end{enumerate}

\textsc{Задача 1.}
Пусть есть инвестор с капиталом $w$ и он может вложить деньги в $2$ неколлериованных $\xi_1 \sim N(\mu_1,\sigma_1^2)$ и $\xi_2 \sim N(\mu_2,\sigma_2^2)$. $u(x) = 1 - e^{ax}$. $\xi_1,\xi_2$ - это доходность, которая выражена в процентах.  В какой пропорции нужно разделить капитал, чтобы максимизировать нашу полезность.

\textsc{Решение}

Введём доли $\alpha$ и $1-\alpha$. Тогда доход будет:
$$s = \alpha w (1+\xi_1) + (1-\alpha)w(1+\xi_2)$$
и нам нужно максимизировать математическое ожидание от функции полезности
$$Eu(s) \to \max  = E (1-e^{\alpha}) = 1 - e^{-wa}Ee^{-wa(\alpha\xi_1+(1-\alpha)\xi_2)}$$

Будем минимизировать величину:
$$Ee^{-wa(\alpha\xi_1+(1-\alpha)\xi_2)} = Ee^{-wa\alpha\xi_1}\cdot  Ee^{-wa(1-\alpha)\xi_2} \to \min$$
$$Ee^{-wa\alpha\xi_1} = Ee^{\beta \xi_1} = \int\limits e^{\beta x_1}\frac{1}{\sqrt{2\pi}}e^{-\frac{(x-\mu_1)^2}{2\sigma_1^2}}dx$$

По производящей функции моментов:
$$e^{\mu\beta + \frac{\beta^2 \sigma_1^2}{2}} = e^{\left(-\mu_1wa\alpha + \frac{w^2 a^2\alpha^2 \sigma_1^2}{2}-\mu_2wa(1-\alpha) + \frac{w^2 a^2(1-\alpha)^2 \sigma_2^2}{2}\right)} \min \alpha$$
$$-wa\mu_1 + w^2a^2\alpha\sigma_1^2 + wa\mu_2-w^2a^2(1-\alpha)\sigma_2^2 = 0$$
$$\alpha = \frac{\mu_1-\mu_2+wa\sigma_2^2}{wa\sigma_1^2+wa\sigma_2^2}$$

Пусть $\mu_1=\mu_2: \alpha = \frac{\sigma_2^2}{\sigma_1^2+\sigma_2^2}$. В какой пропорции нужно разделить, чтобы минимизировать дисперсию.
$$D(w\alpha(1+\xi_1))+D(w(1-\alpha)(1+\xi_2)) = w^2 \alpha^2 D\xi_1 + w^2 (1-\alpha)^2 D\xi_2 = w^2\alpha^2\sigma_1^2 + w^2(1-\alpha)^2\sigma_2^2$$
$$2\alpha w^2\sigma_1^2 -2w^2(1-\alpha)\sigma_2^2 = 0$$
$$\alpha = \frac{\sigma_2^2}{\sigma_1^2+\sigma_2^2}$$

Решение, максимизирующее полезность соответствует решению, минимизирующую диспесию портфеля. Совпадения есть в случае $\mu_1 = \mu_2$. Интересно посмотреть через призму полезности.

Активы, различные портфели, ожидание и дисперсия. $\mu,\sigma$ - спектр доходности. Почему он определяется выпуклой фигурой.

Стандартное отклонение - выпуклая функция. Если есть возможность выбрать из различных портфелей, то мы можем сформировать любой портфель. Данное множество - плотно , сплошное (говорим про овал). 

Такое множество называется \textsc{эффективное множество}. Набор не даёт конкректный портфель, потому что мы не понимаем в чём разница между портфелями. \textbf{Критерий выбора} - поиск точки, в которой полезность максимальная - на кривой выбирает тот портфель, который дает максимальная полезность и в этом случае будет достигаться баланс между двумя теориями.

\textsc{HT}: кривые безразличия - те портфели, между которыми клиент индеферентен. В осях $\mu,\sigma$ и если рассмотреть одинаково полезные портфели, то они будут образовывать выпуклую кривую. И тогда решение - это точка касательной множества всех портфелей и совпадать с базовыми теорями кривых безразличия.

\end{document}