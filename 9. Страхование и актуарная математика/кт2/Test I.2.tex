\documentclass[12pt]{article}
 \usepackage[margin=1in]{geometry} 
\usepackage{amsmath,amsthm,amssymb,amsfonts, enumitem, fancyhdr, color, comment, graphicx, environ, fancyhdr}
\pagestyle{fancy}
\usepackage{cmap}					% поиск в PDF
\usepackage[T2A]{fontenc}			% кодировка
\usepackage[utf8]{inputenc}			% кодировка исходного текста
\usepackage[english,russian]{babel}	% локализация и переносы
\setlength{\parindent}{0pt}
\setlength{\headheight}{35pt}

\lhead{Insurance \& Actuarial Mathematics}
\chead{Test I.2}
\rhead{СПбГЭУ 2020 Радионов А.В.}


\begin{document}

\begin{enumerate}
\setlength\itemsep{-0.15em}

\item В модели коллективного риска, где количество убытков определяется Пуассоновской случайной величиной, функция распределения суммарного ущерба в точке ноль:
\begin{enumerate}
\setlength\itemsep{-0.15em}
    \item делает скачок;
    \item непрерывна;
    \item не определена;
    \item ни одно из вышеперечисленных утверждений не является всегда верным.
\end{enumerate}

\item Рассматривается модель, в которой с вероятностью $p$ реализуется один ущерб, а с вероятностью $(1-p)$ два ущерба. Ущербы независимы и определяются нормальными случайными величинами с ожиданием 5 и дисперсией 1. Плотность распределения общего убытка приведена на рисунке. Тогда:

\begin{figure}[!h]
    \begin{center}
    \includegraphics[scale=0.5]{Tests/I.jpg}
    \end{center}
\end{figure}


\begin{enumerate}
\setlength\itemsep{-0.15em}
    \item $p>0.5$;
    \item $p<0.5$;
    \item $p=0.5$;
    \item ни одно из вышеперечисленных утверждений не является всегда верным.
\end{enumerate}

\item Рассматривается источник риска, ущерб по которому возникает с вероятностью 0.1. В случае возникновения размер ущерба определяется случайной величиной с плотоностью $f_\xi(x)=ax^2,$ $x\in[0;10].$ В противном случае ущерб равен нолю. Найти функцию распределения ущерба и дисперсию ущерба.

\item Рассматривается модель коллективного риска $S_{coll}=\sum\limits_{i=1}^N\xi_i,$ где производящая функция вероятностей числа убытков есть $\phi(z)=\left(\dfrac{0.5z}{1-0.5z}\right)^5,$ а производящая функция моментов $\xi_i$ есть $\psi_{\xi_i}(t)=(1-2t)^{-2.5}.$ Найти $P(N=0),$ $P(N=1),$ $E\xi_i,$ $DS_{coll}.$

\item 
Выберите верные утверждения:
\begin{enumerate}
\setlength\itemsep{-0.15em}
    \item если обе компоненты вектора возвести в квадрат, соответствующая ему копула не изменится;
    \item если обе компоненты вектора возвести в куб, соответствующая ему копула не изменится;
    \item если обе компоненты вектора возвести в квадрат, соответствующая ему копула может не изменится, но может и измениться;
    \item если обе компоненты вектора возвести в куб, соответствующая ему копула может не изменится, но может и измениться.

\end{enumerate}

\item Найти копулу и маргинальные распределения для вектора с совместной функцией распределения $F(x_1,x_2)=\dfrac{x_2^2\sqrt{x_1}/4}{\sqrt{x_1}+x_2^2/4-x_2^2\sqrt{x_1}/4},x_1\in[0;1],x_2\in[0;2].$

 \item Рассматривается двумерный вектор $\zeta,\xi$ с копулой $C(u_1,u_2).$ Найти копулу для вектора $(\vartheta,-\zeta).$

\item Выберите верные утверждения:
\begin{enumerate}
\setlength\itemsep{-0.15em}
    \item максимум из трёх независимых, одинаково распределенных экспоненциальных случайных величин имеет распределение Гумбеля;
    \item максимум из трёх независимых, одинаково распределенных экспоненциальных случайных величин имеет распределение Вейбулла;
    \item максимум из трёх независимых, одинаково распределенных экспоненциальных случайных величин имеет распределение Фреше;
    \item ни одно из вышеперечисленных утверждений не является верным.
\end{enumerate}

\item Рассматривается случайная величина $Z=\max(X,Y),$ случайные величины $X,Y$ независимы. Объясните, как класс предельного распределения для $Z$ будет зависеть от классов пределельных распределений для $X$ и для $Y.$

\item Объясните, какие подходы можно использовать для оценки поведения хвостов распределений с помощью теории экстремальных значений.

\end{enumerate}

\end{document}
