\documentclass[%
12pt, %
final, % 
oneside, % 
onecolumn, %  
centertags]{article} % относится к классу article и размер шрифта 12 пунктовб, {article: статья, report: отчеты и диссертации, book: книга, letter: письмо}

% ------ page construction 

\topmargin= -30pt % насколько сверху будет страница
\textheight= 650pt

% ------ Пакеты расширения

\usepackage[utf8]{inputenc} % задает кодировку, utf-8 кодировка, включающая в себя знаки почти всех языков мира
\usepackage[english, russian]{babel} % подключает необходимые языки, основным языком является английский
\selectlanguage{russian} % настройки будут на английском, но писать будет на русском

\usepackage{euscript}
\usepackage{supertabular}

\usepackage[colorlinks=true,linkcolor=black,unicode=true,urlcolor = blue]{hyperref} %hypered
\usepackage[pdftex]{graphicx} % для графики

\usepackage{amsthm, amssymb, amsmath, amsfonts} % математический пакет, математические шрифты
\usepackage{textcomp}
\usepackage[noend]{algorithmic}
\usepackage[ruled]{algorithm}
\usepackage{lipsum}
\usepackage{indentfirst}
\usepackage{babel}
\usepackage{pgfplots}
\usepackage{setspace}
\usepackage{xcolor}
\usepackage{hyperref}

\linespread{1.2} 
\setlength{\parindent}{2.4em}
\setlength{\parskip}{0.1em}

\pgfplotsset{compat=1.9}
\pgfplotsset{model/.style = {blue, samples = 100}} 
\pgfplotsset{experiment/.style = {red}}

\theoremstyle{plain}
\binoppenalty=10000

\newtheorem{theorem}{Теорема}[section] % theorem

\theoremstyle{definition}
\newtheorem{definition}{Определение}[subsection]

\theoremstyle{remark}
\newtheorem{remark}{Замечание}[section]

\newtheorem{corollary}{Следствие}

\newtheorem{solution}{Решение}

\newtheorem{proposition}{Proposition}

\newtheorem{example}{Пример}

\newtheorem{lemma}{Лемма}[section]

\renewcommand*{\proofname}{Proof}

\graphicspath{ {./image/} }

\usepackage{enumitem}
\begin{document}

\begin{titlepage} 
\begin{center}
\textbf{}\\[10.0cm]
\textbf{\LARGE Страхование и актуарная математика}\\[0.5cm]
\textbf{\Large Test 1 Try 1 23.09.2020} \\[0.2cm]


\begin{center} \large
{Преподаватель:} \\[0.5cm]
\textsc {Радионов Андрей Владимирович}\\
\end{center}

\vfill 



{\large {Александр Широков, ПМ-1701}} \par
{\large {Санкт-Петербург}} \par
{\large {2020 г., 7 семестр}} 

\end{center} 
\end{titlepage}

\newpage

1. Выберите верные утверждения:

\begin{enumerate}
	\item Если предложить несклонному к риску человеку с капиталом 100 и подверженному равномерному на отрезке [0;10] ущербу купить страховой полис с полным покрытием за 5 денежных единиц, он согласится. \textbf{Ответ:} верно.
	\item У несклонного к риску человека коэффициент Эрроу-Пратта может быть как
положительным, так и отрицательным. \textbf{Ответ:} неверно, коэффициент Эрроу-Пратта $r = -\frac{u''(w)}{u'(w)}$, у несклонного к риску человеку функция полезности вогнута, следовательно, $u''(w) <0, u'(w) > 0$, коэффициент положительный.
	\item Человек с линейной функцией полезности не представляет интереса для букмекера. \textbf{Ответ:} верно, так как с линейной функцией полезности человек относится к Risk Neutral.
	\item Если у двух человек одинаковый капитал, но функция полезности первого на три больше, чем у второго, то первый будет готов заплатить за участие в игре больше, чем второй. \textbf{Ответ:} неверно, связано с линейным преобразованием функции полезности (всегда можно сдвинуть в начало координат).
	\item Человек с функцией полезности $u(x) = \sqrt[3]{x}$ склонен к риску. \textbf{Ответ:} человек склонен к риску, когда его функция полезности выпукла - Risk Loving. Функция $u(x) = \sqrt[3]{x}$ является выпуклой - человек склонен к риску. Верно.
\end{enumerate}

\newpage

2. Рассматривается человек с функцией полезности $u(x) = 1- 0.01x^2$, капиталом 50 и
потенциальным ущербом, равномерно распределенном на отрезке $[0;10]$ ($\xi \sim U[0,10])$. Напишите
выражение, с помощью которого можно определить, за сколько можно продать ему:
\begin{enumerate}
	\item страховой полис с полным покрытием
	\item полис, по которому страховая организация оплачивает 50\% убытка.
\end{enumerate}

\textbf{Решение 2.1:} Обозначим за $a$ - сумму, которую готов заплатить человек за полное покрытие. необходимо решить следующее уравнение:
$$u(w-a) = 1- 0.01(50-a)^2 = Eu(w-\xi) = \int\limits_0^{10} u(w-x) f_{\xi}(x)dx = \int\limits_0^{10} (1- 0.01(50-x)^2) \frac{1}{10-0} dx$$
$$ 1- 0.01(50-a)^2 = \int\limits_0^{10} \frac{ 1- 0.01(50-x)^2}{10} dx $$

Это решается.

\textbf{Решение 2.2:}

Для этого небходимо написать следующее выражение:
$$Eu(w-\xi) = Eu(w-a - \frac{\xi}{2})$$
$$\int\limits_0^{10} \frac{ 1- 0.01(50-x)^2}{10} dx = \int\limits_0^{10} \frac{1- 0.01(50-a - \frac{x}{2})^2}{10} dx$$
\newpage

3. Рассматривается человек с функцией полезности $u(x)$. Предложите меру риска,
которая упорядочивала бы случайные величины согласованно с предпочтениями,
порожденными функцией полезности. Объясните, почему использовать такую меру было бы на самом деле не очень осмысленно.

\textbf{Решение:}

В качестве меры риска можно использовать точку максимума (минимума) выпуклой (вогнутой функции). Не очень осмысленно использовать, потому что не учитываются случайные колебания относительно данной точки.

\newpage 

4. На графиках представлены функции распределения двух убытков (левее ноля обе
функции распределения равны нолю).

Выберите верные увтерждения:

\begin{enumerate}[label=\alph*.]
\setlength\itemsep{-0.15em}
    \item $VaR_{70\%}(\xi_1)<VaR_{70\%}(\xi_2).$ \textbf{Ответ:} верно.
    \item $VaR_{70\%}(\xi_1)<VaR_{90\%}(\xi_1).$ \textbf{Ответ:} верно.
    \item $CVaR_{70\%}(\xi_1)<CVaR_{70\%}(\xi_2).$ \textbf{Ответ:} верно.
    \item $CVaR_{70\%}(\xi_2)<CVaR_{90\%}(\xi_2).$ \textbf{Ответ:} верно.
\end{enumerate}

\newpage

5. Выберите верные утверждения

\begin{enumerate}[label=\alph*.]
\setlength\itemsep{-0.15em}
    \item $VaR_{90\%}(\xi+\eta)\leqslant VaR_{90\%}(\xi)+VaR_{90\%}(\eta),$ если $\xi,\eta$ - нормальные случайные величины с ненулевым коэффициентом корреляции.  \textbf{Ответ:} верно, доказывали на паре.
    \item $CVaR_{90\%}(\xi+\eta)\leqslant CVaR_{90\%}(\xi)+CVaR_{90\%}(\eta),$ если $\xi,\eta$ - независимые случайные величины. \textbf{Ответ:} верно.
    \item $VaR_{90\%}(\xi+\eta)= VaR_{90\%}(\xi)+VaR_{90\%}(\eta),$ если ранговый коэффициент корреляции Спирмена равен 1. \textbf{Ответ:} верно, если подставить 1, то под корнем будет $(\sigma_1+\sigma_2)^2$, откуда и получается равенство.
    \item Стандартное отклонение - положительно однородная мера риска. \textbf{Ответ:} свойство положительной однородности меры: $\rhpo(\lambda \xi) = \lambda \rho(\xi)$, $\lambda >0$. Для стандартного отклонения есть свойство, что $\sigma(\lambda \xi) = \vert \lambda \vert \sigma(\xi)$ и при положительном $\lambda > 0$ это выполнено. Верно. 

\end{enumerate}

\newpage

6. Объясните, в чём преимущества и недостатки использования субаддитивной меры
риска.

\textbf{Объяснение:} пусть задана какая-то мера риска $\rho: \Sigma \to R$. Тогда выражение:
$$\rho(\xi + \eta) \leqslant \rho(\xi) + \rho(\eta)$$
называется свойством субаддитивности меры. В чём преимущество использования меры с субаддитивностью? Это мера риска поддаётся логическим сооражениям о адекватном хранении ресурсов. Например, предположим, что обороняются два участка на войне и весьма маловероятно, что на оба участка одновременно нападут противники, так как обычно готовятся атаки по одному пункту. Поэтому мы можем перевести часть оборонительных резервов на другой участок. Таким образом резерв, необходимый для обороны участков вместе меньше, чем резерв, необходмый для обороны участков по отдельности. Таким образом с помощью данной меры риска мы можем высвобождать ресурсы. Из недостатков можно выделить следующее - все данные меры риска придумаются для уменьшения риска за счет повышениея степени его распределенности, но повышение распределенности риска работает в том случае, если бумаги независимы - в этом и недостаток субаддитивной меры.

\newpage

7. Рассматривается выборка из убытков за 200 периодов наблюдений, при этом 150
периодов убыток был равен нолю, 30 периодов убыток был равен 1, 10 периодов
убыток был равен 2, 7 периодов убыток был равен 3, 1 раз убыток был равен 5, 1
раз 7 и 1 раз 15. Оцените непараметрически $\operatorname{VaR}_{90\%}(\xi)$ и $\operatorname{CVaR}_{90\%}(\xi)$.

\textbf{Решение:}

Оценим по частотам соответствующие вероятности выпадения убытков. Они соответственно равны: $0.75, 0.15, 0.05, 0.035, 0.005, 0.005, 0.005$. Тогда по определению посчитаем необходимые величины:

$$\operatorname{VaR}_{90\%}(\xi) = \operatorname{inf}\{x_{\alpha}: P(\xi \geqslant x_{\alpha}) \leqslant 0.01\} = 1$$
$$p_{\operatorname{total}} = 0.15 + 0.05 + 0.035 + 0.005 + 0.005 + 0.005 = 0.25$$
$$\operatorname{CVaR}_{90\%}(\xi) = \mathbb{E}(\xi \vert \xi \geqslant \operatorname{VaR}_{90\%}(\xi)) =  \mathbb{E}(\xi \vert \xi \geqslant 1 = 1 \cdot \frac{ 0.15}{0.25} + 2 \cdot \frac{ 0.05}{0.25} + 3 \cdot \frac{ 0.035}{0.25} +$$
$$ + 5 \cdot \frac{ 0.005}{0.25} + 7 \cdot \frac{ 0.005}{0.25} + 15 \cdot \frac{ 0.005}{0.25} = 1.96$$




\end{document}