\RequirePackage{ifluatex}
\let\ifluatex\relax

\documentclass[aps,%
12pt,%
final,%
oneside,
onecolumn,%
musixtex, %
superscriptaddress,%
centertags]{article} %% 
\topmargin=-40pt
\textheight=650pt
\usepackage[english,russian]{babel}
\usepackage[utf8]{inputenc}
%всякие настройки по желанию%
\usepackage[colorlinks=true,linkcolor=black,unicode=true]{hyperref}
\usepackage{euscript}
\usepackage{supertabular}
\usepackage[pdftex]{graphicx}
\usepackage{amsthm,amssymb, amsmath}
\usepackage{textcomp}
\usepackage[noend]{algorithmic}
\usepackage[ruled]{algorithm}
\usepackage{lipsum}
\usepackage{indentfirst}
\usepackage{babel}
\usepackage{pgfplots}
\usepackage{setspace}
\linespread{1.2}
\pgfplotsset{compat=1.9}
\selectlanguage{russian}
\pgfplotsset{model/.style = {blue, samples = 100}}
\pgfplotsset{experiment/.style = {red}}
\theoremstyle{plain}
\binoppenalty=10000
\newtheorem{theorem}{Теорема}[section] %
\setlength{\parindent}{2.4em}
\setlength{\parskip}{0.1em}
\theoremstyle{definition}
\newtheorem{definition}{Определение}[subsection]
\theoremstyle{remark}
\newtheorem{remark}{Замечание}[section]

\newtheorem{corollary}{Следствие}
\newtheorem{proposition}{Proposition}
\newtheorem{example}{Пример}
\renewcommand*{\proofname}{Proof}

\newtheorem{lemma}{Лемма}[section]

\graphicspath{ {./image/} }
\usepackage{xcolor}
\usepackage{hyperref}


\begin{document}

\begin{titlepage} 
\begin{center}
\textbf{}\\[10.0cm]
\textbf{\LARGE Методы анализа данных}\\[0.5cm]
\textbf{\Large Александр Широков ПМ-1701} \\[0.2cm]


\begin{center} \large
{Преподаватель:} \\[0.5cm]
\textsc {Ивахненко Дарья Александровна}\\
\end{center}

\vfill 



{\large {Санкт-Петербург}} \par
{\large {2020 г., 7 семестр}}
\end{center} 
\end{titlepage}

% Table of contents
\begin{thebibliography}{3}
  \bibitem{A}
\end{thebibliography}
\tableofcontents
\newpage

\section{01.09.2020}

\subsection{Задача обучения по предедентам}

Пусть $X$ - множество объектов, а $Y$ - множество ответов. $y: X \to Y$ - неизвестная зависимость.

Дано: $\{x_1,\ldots,x_l\} \subset X$ - обучающая выборка, а $y_i = y(x_i), i = 1, \ldots,l$ - известные ответы.

Требуется найти $a: X \to Y$ - алгоритм, решающую функцию, приближающую $y$ на всем множестве $X$.

\subsection{Типы задач}

\textbf{Задачи восстановления регрессии:}

\begin{itemize}
	\item $Y = \mathbb{R}$ - вся числовая ось:
	\begin{itemize}
		\item определение температуры воздуха метеорологического поля
		\item оценка влияния факторов потребления
	\end{itemize}
	\item $Y \in [0;+\infty)$:
	\begin{itemize}
		\item задачи медицинской диагностики: прогнозирование ожидаемого время действия препарата
		\item задачи кредитного скоринга: определение величины кредитного лимита
		\item определение расхода топлива по техническим характеристикам
	\end{itemize}
	\item $Y \in [0,1,\ldots,+\infty)$ - счетная целевая переменная
\end{itemize}

\textbf{Задача классификации:}

\begin{itemize}
	\item $Y = \{-1,+1\}$ - классификация на два класса:
	\begin{itemize}
		\item задачи кредитного скоринга: решение о выдаче кредита
		\item предсказание оттока клиентов
	\end{itemize}

	\item $Y = \{1,\ldots,K\}$ - классификация на $K$ непересекающихся классов:
	\begin{itemize}
		\item задачи медицинской диагностики: определение диагноза
		\item распознавание символов
		\item определение жанра
	\end{itemize}

	\item $Y = \{0,1\}^K$ - на $K$ классов, которые могут пересекаться:
	\begin{itemize}
		\item определение ключевых слов для оптимизации поиска
		\item определение присутствующих на фото объектов
	\end{itemize}
\end{itemize}

\textbf{Типы признаков}

\begin{itemize}
	\item $D_j = \{0,1\}$ - бинарный признак $f_j$:
	\begin{itemize}
		\item пол
		\item является ли..?
	\end{itemize}
	\item $|D_j| < \infty$ - номинальный признак $f_j$:
	\begin{itemize}
		\item город
		\item цвет
	\end{itemize}
	\item $|D_j| < \infty$, $D_j$ - упорядочено - порядковый признак $f_j$:
	\begin{itemize}
		\item уровень холестерина (ниже нормы, норма, выше нормы)
	\end{itemize}
	\item $D_j = \textbf{R}$ - количественный признак $f_j$:
	\begin{itemize}
		\item длина и ширина объекта
	\end{itemize}
\end{itemize}


\end{document}