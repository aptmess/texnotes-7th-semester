\documentclass[%
10pt, %
final, % 
oneside, % 
onecolumn, %  
centertags]{article} % относится к классу article и размер шрифта 12 пунктовб, {article: статья, report: отчеты и диссертации, book: книга, letter: письмо}

% ------ page construction 

\topmargin= -30pt % насколько сверху будет страница
\textheight= 650pt

% ------ Пакеты расширения

\usepackage[utf8]{inputenc} % задает кодировку, utf-8 кодировка, включающая в себя знаки почти всех языков мира
\usepackage[english, russian]{babel} % подключает необходимые языки, основным языком является английский
\selectlanguage{russian} % настройки будут на английском, но писать будет на русском

\usepackage{euscript}
\usepackage{supertabular}

\usepackage[colorlinks=true,linkcolor=black,unicode=true,urlcolor = blue]{hyperref} %hypered
\usepackage[pdftex]{graphicx} % для графики

\usepackage{amsthm, amssymb, amsmath, amsfonts} % математический пакет, математические шрифты
\usepackage{textcomp}
\usepackage[noend]{algorithmic}
\usepackage[ruled]{algorithm}
\usepackage{lipsum}
\usepackage{indentfirst}
\usepackage{babel}
\usepackage{pgfplots}
\usepackage{setspace}
\usepackage{xcolor}
\usepackage{hyperref}

\linespread{1.2} 
\setlength{\parindent}{2.4em}
\setlength{\parskip}{0.1em}

\pgfplotsset{compat=1.9}
\pgfplotsset{model/.style = {blue, samples = 100}} 
\pgfplotsset{experiment/.style = {red}}

\theoremstyle{plain}
\binoppenalty=10000

\newtheorem{theorem}{Теорема}[section] % theorem

\theoremstyle{definition}
\newtheorem{definition}{Определение}[subsection]

\theoremstyle{remark}
\newtheorem{remark}{Замечание}[section]

\newtheorem{corollary}{Следствие}

\newtheorem{proposition}{Proposition}

\newtheorem{example}{Пример}

\newtheorem{lemma}{Лемма}[section]

\renewcommand*{\proofname}{Proof}

\graphicspath{ {./image/} }


\begin{document}

\begin{titlepage} 
\begin{center}
\textbf{}\\[10.0cm]
\textbf{\LARGE Планирование расписаний и управление доходами}\\[0.5cm]
\textbf{\Large Александр Широков ПМ-1701} \\[0.2cm]


\begin{center} \large
{Преподаватель:} \\[0.5cm]
\textsc {Васильев Юрий Михайлович}\\
\end{center}

\vfill 



{\large {Санкт-Петербург}} \par
{\large {2020 г., 7 семестр}}
\end{center} 
\end{titlepage}

% Table of contents
\begin{thebibliography}{3}
  \bibitem{A}
\end{thebibliography}
\tableofcontents
\newpage

\section{02.09.2020}

\subsection{Задача из авиакомпании Россия}

В задачах планирования авиаперелетов:

\begin{itemize}
	\item расписание судов
	\item маршутизация
	\item построение графика полета летного состава
\end{itemize}

Мы поговорим о построении графика полета летного состава. Зарплата бортпроводника зависит от навыков и от некоторыз других факторов, но значительная часть денег тратилась на штрафы, которые выплачивались в пользу бортпроводников, потому что есть \textit{приказ}, о котором бортпроводник не может проводить в воздухе больше определенного времени в воздухе. 

Расписание в авиакомпании Россия делалось вручную и компания тратила много денег на выплаты бортпроводникам. OpenSky - программное обеспечение для обслуживания бортпроводников, но оно использовалось.

Множество борпроводников разбито на $4$ подмножеств с примерно одинаковыми характеристиками. Каждое подмножество называется \textbf{книга}.

\textbf{Рейс} - перелет из Петербурга в Москву, а \textbf{связка} - перелет из Петербурга в Мосвку и обратно. 

Множесто связок разбивалось на $4$ подмножества.

После этого соединяется первая книга и первый рейс и получается \textbf{рабочий стол}. Каждый рабочий стол можно описать характеристиками какими-то. С каждым рабочим столом работает один эксперт и все оказываются без перегрузов.

Задача: необходимо так разбить связки на подмножества, чтобы характеристики рабочих столов были примерно одинаковы.

\subsubsection{BiWay (ToWay) Number Partitional Problem}

Дано $n$ натуральных чисел и мультимножество (элементы могут повторяться) $S$, которое описывает этот набор $n$. Нам необходимо разбить подмножество $S=\{s_1,\ldots,s_n\}$ на два подмножества, каждое подмножество характерирузет сумму чисел, чтобы минимизировать максимальную сумму чисел в подмножестве.

\textbf{Greedy alghorytm}

\begin{enumerate}
	\item Отсортировать $S$ в порядке убывания
	\item На каждом шаге мы последовательно распределяем в две группы, кладём в группу с текущей наименьшей суммой. Если сумма одинакова, то кладем случайно.
\end{enumerate}

\textbf{Complete Greedy Alghorytm}

\begin{enumerate}
	\item Сортируем мультимножество в порядке убывания (распределяем большие числа и добиваем маленькими)
	\item Данный алгоритм исследует бинарное дерево, где каждому уровень - число из сортированного мультимножества, в каждой вершине - ветвление. В левой ветке - кладем в группу с наименьшей суммой, а в правой ветке - с наибольшей.
\end{enumerate}

Правила, позволяющие сократить размер нашего дерева:

\begin{itemize}
	\item Если сумма чисел в подмножествах равна, то мы кладем число только в одно подмножество
	\item Если оставшиеся распределенные числа не превосходят разницу между подмножествами, то мы кладем эти числа в группу с наименьшей суммой.
\end{itemize}

Домашнее задание: реализация алгоритма, причем настрока алгоритма в трех вариантах:
\begin{itemize}
	\item Исследует полное дерево решений и находит ответ;
	\item Алгоритм работает заданное число секунд и возвращает наилучший найденный результат за $t$ время (рекурсивная функция(оставшиеся числа, подмножества 1, подмножества 2))
	\item Первое найденное решение (первый лист, который мы нашли).
\end{itemize}

\textbf{Алгоритм Кармаркара-Карпа (эвристический)}

\begin{enumerate}
	\item Сортируем мультимножество в порядке убывания (распределяем большие числа и добиваем маленькими)
	\item Два наибольших числа заменяется на их разницу и кладём эту разницу в список с сортировкой и опять пересортировываем - кладем числа в два разных подмножества (интерпретация).

	\item Так делаем, пока не получим одно число: разницу межде максимальным и минимальным подмножеством
	\item Восстанавливаем
\end{enumerate}

Пример:
$$\{16,15,12,10,5,1\} \mapsto \{12,10,5,1,1\} \mapsto \{5,2,1,1\} \mapsto \{3,1,1\} \mapsto \{2,1\} \mapsto \{1\}$$

\textbf{Compete алгоритм Кармакара-Карпа}

\begin{enumerate}
	\item Сортируем мультимножество в порядке убывания (распределяем большие числа и добиваем маленькими)
	\item Исследуем бинарное дерево в глубину, исследуя левую ветку
\end{enumerate}

Домашнее задание: реализовать алгоритм для решения.

\subsubsection{MultiWay (ToWay) Number Partitional Problem}

Дано $n$ натуральных чисел и мультимножество (элементы могут повторяться) $S$, которое описывает этот набор $n$. Нам необходимо разбить подмножество $S=\{s_1,\ldots,s_n\}$ на $K$ подмножества, каждое подмножество характерирузет сумму чисел, чтобы минимизировать:
\begin{enumerate}
	\item минимизация максимальной суммы
	\item максимизация минимальной суммой
	\item минимизация разности между наибольшей и наименьшей суммой в подмножествах
	\item идеальная сумма - $\frac{S}{K}$ - минимизировать отклонения идеальной суммы
\end{enumerate}

$$X_{i,j} = \begin{cases}
	1, \text{if } S_i \text{ in } j \quad S_j \\
	0
\end{cases}$$

$$Z-W \to \min$$
$$\sum\limits_{j=1}^k X_{s,j}=1 \quad \forall s \in S$$

$Z$ - наибольшая сумма через $x$, а $W$ - наименьшую сумму через подмножества

$$Z \geq \sum\limits_{i=1}^n s_i X_{i,j} \quad \forall j \in \{1,\ldots,k\}$$
$$W \leq \sum\limits_{i=1}^n s_i X_{i,j} \quad \forall j \in \{1,\ldots,k\}$$
$$X_{i,j} \in \{0,1\}, \quad \forall i \in \{1,\ldots,n\}, \forall j \in \{1,\ldots,k\}$$

\textbf{Жадный алгоритм}

$L_j (S_1,S_2,\ldots,S_k,S_i)$

данная функция возвращает значение целевой функции, если мы положим число $S_i$ в $j$-е подмножество.

На каждом шаге алгоритма мы ищем такое $j \in \{1,\ldots,k\}$ такое, что значение целевой функции $gr = argmin L_j$ и так до тех пор пока мы не распределим все наши числа из отсортированного подмножества.
$$S_{gr} = S_{gr} \cup \{S_i\}$$

Программирование:

$c$ - список неизвестных, $m$ - коэффициенты при ограничениях, $\{\{const\},\{type\}\}$. Если $0$, то равенство, если $1$, то $\geq$, если $-1$, то $\leq$. 4-ый аргумент - интервалы, в которых могут применять значения неизвестные - $lbound, ubound$. Последний - какому множеству чисел принадлежит тип.

Домашнее задание: минимизация сумма отклонения по модулю от идеального разбиения и реализация.

$$\overline{y} = \frac{\sum S}{K}$$
$$\sum\limits_{i=1}^k |y_i - \overline{y}|$$

Линеаризация

\begin{itemize}
	\item Линеаризация модуля в ограничениях
	$$|X| \leq b (X=\sum\limits_{i=1}^n a_ix_i, b \geq 0)$$
	$$\begin{cases}
		x \leq b \\
		x \geq -b
	\end{cases}$$
	\item Допустимые значения
	$$x = 0 \text{ или } 0 \leq X \leq b, a >0$$
	$$y = \begin{cases}
		0, x=0 \\
		1
	\end{cases}$$
	$$\begin{cases}
		x \geq ay \\
		x \leq by \\
		y \in \{0,1\}
	\end{cases}$$

	\item Условие ИЛИ
	$$\sum\limits_{i=1} ^n c_ix_i \to \min$$
	$$\sum\limits_{i=1}^n a_{1,i}x_i \leq b_1 + M_1y$$
	$$\sum\limits_{i=1}^n a_{2,i}x_i \leq b_2+ M_2(1-y) \quad y \in \{0,1\}$$  
	\item Модуль со знаком $\geq$ 
	\item IF
	$$\text{if} \quad  \sum\limits_{i=1}^n a_{1,i}x_i \leq b_1 \to  \sum\limits_{i=1}^n a_{1,i}x_i \geq b_1 + \varepsilon$$
	$$\text{then} \quad \sum\limits_{i=1}^n a_{2,i}x_i \leq b_2 $$
	и мы превратили в третий пункт
	$$y \in \{0,1\}$$
	\item Умножение бинарных переменных
	$$\ldots + x_1\cdot x_2 + \ldots \leq b$$
	$$x_1,x_2 \in \{0,1\}$$
	$$y \in \{0,1\}$$
	$$y \leq x_1$$
	$$y \leq x_2$$
	$$y \geq x_1+x_2-1$$
	\begin{table}[H]
	\begin{center}
		\begin{tabular}{c|c|c|c|c} 
		$x_1$ & 1 & 0 & 0 & 0  \\ 
		$x_2$ & 1 & 0 & 1 & 0  \\ \hline
		$y$ & 1 & 0 & 0 & 0 
		\end{tabular}

	Линеаризация
	\end{center}
\end{table}
\end{itemize}

\subsection{Multi Dimensional Multi Way NPP}

Будем заниматься векторами. Минимизация максимальной разности по координатам между подмножествами.
$$1: (a_1,a_2,a_3)$$
$$2: (b_1,b_2,b_3)$$
$$3: (c_1,c_2,c_3)$$
$$\max (|a_1-b_1|, |a_2-b_2|,|a_3-b_3|) \to \min$$

Пусть $NC$ - размерность вектора, $NV$ - количество векторов, $NK$ - число групп.

Множество:
$$S = \{s_i \vert s_i = (s_{i,2},s_{i,2},\ldots,s_{i,NC})\}, i \in \{1,\ldots,NV\}$$
Неизвестные:
$$x_{s,k} = \begin{cases}
	1, - s \in NK \\
	0 
\end{cases}$$

Введем дополнительную переменную $y_{c,k}$ - сумма векторов из подмножества $k$ по координате $c$:
$$\max |y_{c,k_1} - y_{c,k_2}| \to \min$$
$$k_1,k_2 \in \{1,\ldots, NK\}$$
$$c \in \{1,\ldots,NC\}$$

Нам нужно найти группу $k_1,k_2$ и $c$ дают разницу по модулю между соответствующими $c$.

Так мы делаем для:
\begin{enumerate}
	\item $$\sigma \geq  y_{c,k_1} - y_{c,k_2}  \quad \forall k_1,k_2 \in \{1,\ldots,NK\}, k_1 < k_2$$
			$$\sigma \geq -y_{c,k_1} + y_{c,k_2} \quad \forall k_1,k_2 \in \{1,\ldots,NK\}, k_1 < k_2$$
			$$\sigma \to \min$$
	\item $$\sum\limits_{k=1}^{NK} x_{s,k} = 1, \qquad \forall s \in S$$ 
	\item $$y_{c,k} = \sum\limits_{s\in S} S_c \cdot x_{s,k} \forall c \in \{1,\ldots,NC\}, k \in\{1,\ldots,NK\}$$
	$$x_{s,k} \in \{0,1\} \forall s \in S, \forall k \in \{1,\ldots,NK\}$$

	Всего незивестынх: $NV*NK +1$
\end{enumerate}

\subsection{Рассмотрим взвешенную сумму средних квадратов отклонений}
$$\sum\limits_{k=1}^{NK}\sum\limits_{c=1}^{NC}w_c\left(1-\frac{y_{c,k}}{\hat{y}_c}\right)^2 \to \min$$
$\hat{y_c}$ - суммируем покоординатно $c$ и делим на $NK$ - идеальное значение по характеристике $c$ в подмножестве.

Чем больше $w_c$, тем больше значит тебя критерий равномерности - тем больше равны координатым векторов.

Такая запись нелинейна по $y$, то на дом будет модуль.

1-е ограничение нужно заменить на связь сигм с дельтами.

Усложним еще задачу.

\subsection{Критерий равномерности}

\begin{enumerate}
	\item Общее число ночных связок
	\item Среднее рабочее время на бортпроводника - берем подмножество связок, попавших на рабочий стол - суммируем время.
\end{enumerate}

Можем обобщить: что каждый вектор $S_{i,c,k}$ имеет разные координаты для разных подгрупп.

Приращение по характеристике $c$ при добавлении $i$ в $k$ подгруппе.
\newpage
\subsection{Minimize Differencse}

\textsc{Входные данные}: 
\begin{enumerate}
	\item $S$ - множество векторов
	\item $NV$ - мощность множества $S$
	\item $NC$ - размерность вектора $s \in S$, то есть каждый вектор описывается $NC$ числовыми координатами
	$$C = \{1,\ldots,NC\}$$

	Множество $S$ задаётся следующим образом:
	$$S = \{s_{i,1},s_{i,2},\ldots,s_{i,NC}\} \quad \forall i \in \{1,\ldots,NV\}$$
	\item $NK$ - число групп:
	$$K = \{1,\ldots,NK\}$$
\end{enumerate}
\textsc{Дополнение к входным данным}:
\begin{itemize}
	\item Введём дополнительную переменную $y_{c,k}$ - суммарное значение координаты $c \in C$ для группы $k \in K$.
\end{itemize}

\textsc{Задача:} необходимо распределить векторы из $S$ по $NK$ группам, причём каждый вектор должен быть представлен в единственной группе.

\textsc{Целевая функция:} минимизация максимальной разницы по модулю между двумя группами по координате среди всех координат и всех групп:
$$\underset{\underset{c\in C}{k_1,k_2\in K}}{\max} \left \vert y_{c,k_1} - y_{c,k_2} \right \vert \to \min$$

\textit{Пояснение:} необходимо найти две группы $k_1$ и $k_2$ и такую координату $c$, которые бы минимизировали максимальную разность.
\newpage
\textsc{Ограничения}:

\begin{enumerate}
	\item Каждый вектор должен находиться строго в одной из групп: 
	$$\sum\limits_{k=1}^{NK}x_{s,k} = 1 : \forall s \in S$$
	$$x_{s,k} \in \{0,1\}$$
	\item Сумма по каждой координате в каждой группе:
	$$y_{c,k} = \sum\limits_{s\in S} s_c \cdot x_{s,k}: \forall c \in C \  \forall k \in K$$
	\item Введём переменную $\sigma$, являющуюся максимальную разность по координате в группах. Её необходимо минимизировать:
	$$\sigma \to \min$$

	Введём ограничение, связывающую $\sigma$ и исходную целевую функцию:
	$$\left\vert y_{c,k_1} - y_{c,k_2} \right\vert \leq \sigma : \forall k_1,k_2 \in K \ \forall c \in C \ k_1 < k_2$$

	Модуль расскрывается через два неравенства:
	$$y_{c,k_1} - y_{c,k_2} - \sigma \leq 0$$
	$$-y_{c,k_1} + y_{c,k_2} - \sigma \leq 0$$ 
\end{enumerate}

Всего в задаче $NV\cdot NK + 1$ переменных.

\newpage

\subsection{Weighted Minimize}

\textsc{Входные данные}: 
\begin{enumerate}
	\item $S$ - множество векторов
	\item $NV$ - мощность множества $S$
	\item $NC$ - размерность вектора $s \in S$, то есть каждый вектор описывается $NC$ числовыми координатами
	$$C = \{1,\ldots,NC\}$$

	Множество $S$ задаётся следующим образом:
	$$S = \{s_{i,1},s_{i,2},\ldots,s_{i,NC}\} \quad \forall i \in \{1,\ldots,NV\}$$
	\item $NK$ - число групп:
	$$K = \{1,\ldots,NK\}$$
\end{enumerate}
\textsc{Дополнение к входным данным}:
\begin{itemize}
	\item Введём дополнительную переменную $y_{c,k}$ - суммарное значение координаты $c \in C$ для группы $k \in K$.
	\item Введём дополнительные \textsc{идеальные} константные значения следующим образом:
	$$\hat{y}_c = \frac{\sum\limits_{i=1}^{NV}s_{i,c}}{NK} : \forall c \in C$$
	\item Введём константы весов, отвечающие за значимость уравнивания по определенной координате - чем больше значение веса, тем важнее уравнивать множество по данной координате:
	$$W = \{w_1,\ldots,w_{NC}\}$$
\end{itemize}

\textsc{Целевая функция:} минимизация взвешенной ($W$) суммы модулей относительных отклонений $y_{c,k}$ от $\hat{y}_c$ по каждой из координат для каждой группы с учётом весов $W$:
$$\sum\limits_{c=1}^{NC} w_c \cdot \sum\limits_{k=1}^{NK} \left\vert 1 - \frac{y_{c,k}}{\hat{y}_c}\right\vert \to \min$$

\newpage
\textsc{Ограничения}:

\begin{enumerate}
	\item Каждый вектор должен находиться строго в одной из групп: 
	$$\sum\limits_{k=1}^{NK}x_{s,k} = 1 : \forall s \in S$$
	$$x_{s,k} \in \{0,1\}$$
	\item Сумма по каждой координате в каждой группе:
	$$y_{c,k} = \sum\limits_{s\in S} s_c \cdot x_{s,k}: \forall c \in C \  \forall k \in K$$
	\item Введём $NC\cdot NK$ переменных $\sigma[c,k]$, являющуюся верхней границей максимальной величины отклонения. Будем минимизировать сумму всех этих переменных:
	$$\sum\limits_{c=1}^{NC} w_c \cdot \sum\limits_{k=1}^{NK} \sigma[c,k] \to \min$$

	Введём следующие ограничения для относительных отклонений:
	$$\left\vert 1 - \frac{y_{c,k}}{\hat{y}_c}\right\vert \leq \sigma[c,k] : \forall c \in C \ \forall k \in K$$

	Модуль расскрывается через два неравенства:
	$$1 - \frac{y_{c,k}}{\hat{y}_c} - \sigma[c,k] \leq 0$$
	$$-1 + \frac{y_{c,k}}{\hat{y}_c} - \sigma[c,k] \leq 0$$ 
\end{enumerate}

Всего в задаче $NV\cdot NK + NC\cdot NK = NK(NV+NC)$ переменных.


\newpage
\subsection{Weighted Choose Minimize}

\textsc{Входные данные}: 
\begin{enumerate}
	\item $S$ - множество векторов
	\item $NV$ - мощность множества $S$
	\item $NC$ - размерность вектора $s \in S$, то есть каждый вектор описывается $NC$ числовыми координатами
	$$C = \{1,\ldots,NC\}$$
	\item $NK$ - число групп:
	$$K = \{1,\ldots,NK\}$$

	Координаты векторов могут отличаться в зависимости от попадания в подмножество, поэтому множество $S$ задаётся следующим образом:
	$$S = \{s_{i,k,1},s_{i,k,2},\ldots,s_{i,k,NC}\} \quad \forall i \in \{1,\ldots,NV\} \quad \forall k \in K$$
	
\end{enumerate}
\textsc{Дополнение к входным данным}:
\begin{itemize}
	\item Введём дополнительную переменную $y_{c,k}$ - суммарное значение координаты $c \in C$ для группы $k \in K$.
	\item Введём дополнительные \textsc{идеальные} константные значения следующим образом:
	$$\hat{y}_{c,k} = \frac{\sum\limits_{i=1}^{NV}s_{i,k,c}}{NK} : \forall c \in C \ \forall k \in K$$
	\item Введём константы весов, отвечающие за значимость уравнивания по определенной координате - чем больше значение веса, тем важнее уравнивать множество по данной координате:
	$$W = \{w_1,\ldots,w_{NC}\}$$
\end{itemize}

\textsc{Целевая функция:} минимизация взвешенной ($W$) суммы модулей относительных отклонений $y_{c,k}$ от $\hat{y}_{c,k}$ по каждой из координат для каждой группы с учётом весов $W$:
$$\sum\limits_{c=1}^{NC} w_c \cdot \sum\limits_{k=1}^{NK}  \left\vert 1 - \frac{y_{c,k}}{\hat{y}_{c,k}}\right\vert \to \min$$

\newpage
\textsc{Ограничения}:

\begin{enumerate}
	\item Каждый вектор должен находиться строго в одной из групп и вектор из группы возможных векторов в зависимости от номера группы должен быть тоже один.
	$$\sum\limits_{i=1}^{NK} \sum\limits_{k=1}^{NK}x_{s,i,k} = 1 : \forall s \in S $$
	$$x_{s,i,k} \in \{0,1\}$$

	Количество переменных: $NV\cdot NK^2$
	\item Сумма по каждой координате в каждой группе:
	$$y_{c,k} = \sum\limits_{i=1}^{NK} \sum\limits_{s\in S} s_{i,c} \cdot x_{s,i,k}:  \forall c \in C \  \forall k \in K $$
	\item Введём $NC\cdot NK$ переменных $\sigma[c,k]$, являющуюся верхней границей максимальной величины отклонения. Будем минимизировать сумму всех этих переменных:
	$$\sum\limits_{c=1}^{NC} w_c \cdot \sum\limits_{k=1}^{NK} \sigma[c,k] \to \min$$

	Введём следующие ограничения для относительных отклонений:
	$$\left\vert 1 - \frac{y_{c,k}}{\hat{y}_{c,k}}\right\vert \leq \sigma[c,k] : \forall c \in C \ \forall k \in K$$

	Модуль расскрывается через два неравенства:
	$$1 - \frac{y_{c,k}}{\hat{y}_{c,k}} - \sigma[c,k] \leq 0$$
	$$-1 + \frac{y_{c,k}}{\hat{y}_{c,k}} - \sigma[c,k] \leq 0$$ 
\end{enumerate}

Всего в задаче $NV\cdot NK^2 + NC \cdot NK = NK\cdot(NV \cdot NK + NC)$ переменных.












\end{document}