\RequirePackage{ifluatex}
\let\ifluatex\relax

\documentclass[aps,%
12pt,%
final,%
oneside,
onecolumn,%
musixtex, %
superscriptaddress,%
centertags]{article} %% 
\topmargin=-40pt
\textheight=650pt
\usepackage[english,russian]{babel}
\usepackage[utf8]{inputenc}
%всякие настройки по желанию%
\usepackage[colorlinks=true,linkcolor=black,unicode=true]{hyperref}
\usepackage{euscript}
\usepackage{supertabular}
\usepackage[pdftex]{graphicx}
\usepackage{amsthm,amssymb, amsmath}
\usepackage{textcomp}
\usepackage[noend]{algorithmic}
\usepackage[ruled]{algorithm}
\usepackage{lipsum}
\usepackage{indentfirst}
\usepackage{babel}
\usepackage{pgfplots}
\usepackage{setspace}
\linespread{1.2}
\pgfplotsset{compat=1.9}
\selectlanguage{russian}
\pgfplotsset{model/.style = {blue, samples = 100}}
\pgfplotsset{experiment/.style = {red}}
\theoremstyle{plain}
\binoppenalty=10000
\newtheorem{theorem}{Теорема}[section] %
\setlength{\parindent}{2.4em}
\setlength{\parskip}{0.1em}
\theoremstyle{definition}
\newtheorem{definition}{Определение}[subsection]
\theoremstyle{remark}
\newtheorem{remark}{Замечание}[section]

\newtheorem{corollary}{Следствие}
\newtheorem{proposition}{Proposition}
\newtheorem{example}{Пример}
\renewcommand*{\proofname}{Proof}

\newtheorem{lemma}{Лемма}[section]

\graphicspath{ {./image/} }
\usepackage{xcolor}
\usepackage{hyperref}


\begin{document}

\begin{titlepage} 
\begin{center}
\textbf{}\\[10.0cm]
\textbf{\LARGE Планирование расписаний и управление доходами}\\[0.5cm]
\textbf{\Large Александр Широков ПМ-1701} \\[0.2cm]


\begin{center} \large
{Преподаватель:} \\[0.5cm]
\textsc {Васильев Юрий Михайлович}\\
\end{center}

\vfill 



{\large {Санкт-Петербург}} \par
{\large {2020 г., 7 семестр}}
\end{center} 
\end{titlepage}

% Table of contents
\begin{thebibliography}{3}
  \bibitem{A}
\end{thebibliography}
\tableofcontents
\newpage

\section{02.09.2020}

\subsection{Задача из авиакомпании Россия}

В задачах планирования авиаперелетов:

\begin{itemize}
	\item расписание судов
	\item маршутизация
	\item построение графика полета летного состава
\end{itemize}

Мы поговорим о построении графика полета летного состава. Зарплата бортпроводника зависит от навыков и от некоторыз других факторов, но значительная часть денег тратилась на штрафы, которые выплачивались в пользу бортпроводников, потому что есть \textit{приказ}, о котором бортпроводник не может проводить в воздухе больше определенного времени в воздухе. 

Расписание в авиакомпании Россия делалось вручную и компания тратила много денег на выплаты бортпроводникам. OpenSky - программное обеспечение для обслуживания бортпроводников, но оно использовалось.

Множество борпроводников разбито на $4$ подмножеств с примерно одинаковыми характеристиками. Каждое подмножество называется \textbf{книга}.

\textbf{Рейс} - перелет из Петербурга в Москву, а \textbf{связка} - перелет из Петербурга в Мосвку и обратно. 

Множесто связок разбивалось на $4$ подмножества.

После этого соединяется первая книга и первый рейс и получается \textbf{рабочий стол}. Каждый рабочий стол можно описать характеристиками какими-то. С каждым рабочим столом работает один эксперт и все оказываются без перегрузов.

Задача: необходимо так разбить связки на подмножества, чтобы характеристики рабочих столов были примерно одинаковы.

\subsubsection{BiWay (ToWay) Number Partitional Problem}

Дано $n$ натуральных чисел и мультимножество (элементы могут повторяться) $S$, которое описывает этот набор $n$. Нам необходимо разбить подмножество $S=\{s_1,\ldots,s_n\}$ на два подмножества, каждое подмножество характерирузет сумму чисел, чтобы минимизировать максимальную сумму чисел в подмножестве.

\textbf{Greedy alghorytm}

\begin{enumerate}
	\item Отсортировать $S$ в порядке убывания
	\item На каждом шаге мы последовательно распределяем в две группы, кладём в группу с текущей наименьшей суммой. Если сумма одинакова, то кладем случайно.
\end{enumerate}

\textbf{Complete Greedy Alghorytm}

\begin{enumerate}
	\item Сортируем мультимножество в порядке убывания (распределяем большие числа и добиваем маленькими)
	\item Данный алгоритм исследует бинарное дерево, где каждому уровень - число из сортированного мультимножества, в каждой вершине - ветвление. В левой ветке - кладем в группу с наименьшей суммой, а в правой ветке - с наибольшей.
\end{enumerate}

Правила, позволяющие сократить размер нашего дерева:

\begin{itemize}
	\item Если сумма чисел в подмножествах равна, то мы кладем число только в одно подмножество
	\item Если оставшиеся распределенные числа не превосходят разницу между подмножествами, то мы кладем эти числа в группу с наименьшей суммой.
\end{itemize}

Домашнее задание: реализация алгоритма, причем настрока алгоритма в трех вариантах:
\begin{itemize}
	\item Исследует полное дерево решений и находит ответ;
	\item Алгоритм работает заданное число секунд и возвращает наилучший найденный результат за $t$ время (рекурсивная функция(оставшиеся числа, подмножества 1, подмножества 2))
	\item Первое найденное решение (первый лист, который мы нашли).
\end{itemize}

\textbf{Алгоритм Кармаркара-Карпа (эвристический)}

\begin{enumerate}
	\item Сортируем мультимножество в порядке убывания (распределяем большие числа и добиваем маленькими)
	\item Два наибольших числа заменяется на их разницу и кладём эту разницу в список с сортировкой и опять пересортировываем - кладем числа в два разных подмножества (интерпретация).

	\item Так делаем, пока не получим одно число: разницу межде максимальным и минимальным подмножеством
	\item Восстанавливаем
\end{enumerate}

Пример:
$$\{16,15,12,10,5,1\} \mapsto \{12,10,5,1,1\} \mapsto \{5,2,1,1\} \mapsto \{3,1,1\} \mapsto \{2,1\} \mapsto \{1\}$$

\textbf{Compete алгоритм Кармакара-Карпа}

\begin{enumerate}
	\item Сортируем мультимножество в порядке убывания (распределяем большие числа и добиваем маленькими)
	\item Исследуем бинарное дерево в глубину, исследуя левую ветку
\end{enumerate}

Домашнее задание: реализовать алгоритм для решения.

\subsubsection{MultiWay (ToWay) Number Partitional Problem}

Дано $n$ натуральных чисел и мультимножество (элементы могут повторяться) $S$, которое описывает этот набор $n$. Нам необходимо разбить подмножество $S=\{s_1,\ldots,s_n\}$ на $K$ подмножества, каждое подмножество характерирузет сумму чисел, чтобы минимизировать:
\begin{enumerate}
	\item минимизация максимальной суммы
	\item максимизация минимальной суммой
	\item минимизация разности между наибольшей и наименьшей суммой в подмножествах
	\item идеальная сумма - $\frac{S}{K}$ - минимизировать отклонения идеальной суммы
\end{enumerate}

$$X_{i,j} = \begin{cases}
	1, \text{if } S_i \text{ in } j \quad S_j \\
	0
\end{cases}$$

$$Z-W \to \min$$
$$\sum\limits_{j=1}^k X_{s,j}=1 \quad \forall s \in S$$

$Z$ - наибольшая сумма через $x$, а $W$ - наименьшую сумму через подмножества

$$Z \geq \sum\limits_{i=1}^n s_i X_{i,j} \quad \forall j \in \{1,\ldots,k\}$$
$$W \leq \sum\limits_{i=1}^n s_i X_{i,j} \quad \forall j \in \{1,\ldots,k\}$$
$$X_{i,j} \in \{0,1\}, \quad \forall i \in \{1,\ldots,n\}, \forall j \in \{1,\ldots,k\}$$

\textbf{Жадный алгоритм}

$L_j (S_1,S_2,\ldots,S_k,S_i)$

данная функция возвращает значение целевой функции, если мы положим число $S_i$ в $j$-е подмножество.

На каждом шаге алгоритма мы ищем такое $j \in \{1,\ldots,k\}$ такое, что значение целевой функции $gr = argmin L_j$ и так до тех пор пока мы не распределим все наши числа из отсортированного подмножества.
$$S_{gr} = S_{gr} \cup \{S_i\}$$

Программирование:

$c$ - список неизвестных, $m$ - коэффициенты при ограничениях, $\{\{const\},\{type\}\}$. Если $0$, то равенство, если $1$, то $\geq$, если $-1$, то $\leq$. 4-ый аргумент - интервалы, в которых могут применять значения неизвестные - $lbound, ubound$. Последний - какому множеству чисел принадлежит тип.

Домашнее задание: минимизация сумма отклонения по модулю от идеального разбиения и реализация.

$$\overline{y} = \frac{\sum S}{K}$$
$$\sum\limits_{i=1}^k |y_i - \overline{y}|$$

Линеаризация

\begin{itemize}
	\item Линеаризация модуля в ограничениях
	$$|X| \leq b (X=\sum\limits_{i=1}^n a_ix_i, b \geq 0)$$
	$$\begin{cases}
		x \leq b \\
		x \geq -b
	\end{cases}$$
	\item Допустимые значения
	$$x = 0 \text{ или } 0 \leq X \leq b, a >0$$
	$$y = \begin{cases}
		0, x=0 \\
		1
	\end{cases}$$
	$$\begin{cases}
		x \geq ay \\
		x \leq by \\
		y \in \{0,1\}
	\end{cases}$$

	\item Условие ИЛИ
	$$\sum\limits_{i=1} ^n c_ix_i \to \min$$
	$$\sum\limits_{i=1}^n a_{1,i}x_i \leq b_1 + M_1y$$
	$$\sum\limits_{i=1}^n a_{2,i}x_i \leq b_2+ M_2(1-y) \quad y \in \{0,1\}$$  
	\item Модуль со знаком $\geq$ 
	\item IF
	$$\text{if} \quad  \sum\limits_{i=1}^n a_{1,i}x_i \leq b_1 \to  \sum\limits_{i=1}^n a_{1,i}x_i \geq b_1 + \varepsilon$$
	$$\text{then} \quad \sum\limits_{i=1}^n a_{2,i}x_i \leq b_2 $$
	и мы превратили в третий пункт
	$$y \in \{0,1\}$$
	\item Умножение бинарных переменных
	$$\ldots + x_1\cdot x_2 + \ldots \leq b$$
	$$x_1,x_2 \in \{0,1\}$$
	$$y \in \{0,1\}$$
	$$y \leq x_1$$
	$$y \leq x_2$$
	$$y \geq x_1+x_2-1$$
	\begin{table}[H]
	\begin{center}
		\begin{tabular}{c|c|c|c|c} 
		$x_1$ & 1 & 0 & 0 & 0  \\ 
		$x_2$ & 1 & 0 & 1 & 0  \\ \hline
		$y$ & 1 & 0 & 0 & 0 
		\end{tabular}

	Линеаризация
	\end{center}
\end{table}
\end{itemize}



\end{document}